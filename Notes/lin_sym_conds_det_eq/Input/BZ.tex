We remind ourselves that we want to study the BZ model

\begin{equation*}
  \begin{split}
    \dv{u}{t}&=\dfrac{1}{\varepsilon}v-\dfrac{1}{\varepsilon}\left(\dfrac{1}{3}u^3-u\right),\\
    \dv{v}{t}&=-u.\\    
    \end{split}
\end{equation*}
and we are looking for a generator of the following kind:
\begin{equation}
X=\xi(t,u,v)\partial_t+\eta_1(t,u,v)\partial_u+\eta_2(t,u,v)\partial_v.
\end{equation}
The linearised symmetry condition 1 is formulated as follows
\begin{align}
  - u \frac{d}{d v} \eta_1 + \pdv{\eta_1}{t} + \frac{\eta_1 u^{2}}{\varepsilon} - \frac{\eta_1}{\varepsilon} - \frac{\eta_2}{\varepsilon} - \frac{u^{4} \frac{d}{d v} \xi}{3 \varepsilon}&\nonumber\\
  - \frac{u^{3} \frac{d}{d u} \eta_1}{3 \varepsilon} + \frac{u^{3} \pdv{\xi}{t}}{3 \varepsilon} + \frac{u^{2} \frac{d}{d v} \xi}{\varepsilon} + \frac{u v \frac{d}{d v} \xi}{\varepsilon}&\nonumber\\
  + \frac{u \frac{d}{d u} \eta_1}{\varepsilon} - \frac{u \pdv{\xi}{t}}{\varepsilon} + \frac{v \frac{d}{d u} \eta_1}{\varepsilon} - \frac{v\pdv{\xi}{t}}{\varepsilon}&\nonumber\\
  - \frac{u^{6} \frac{d}{d u} \xi}{9 \varepsilon^{2}} + \frac{2 u^{4} \frac{d}{d u} \xi}{3 \varepsilon^{2}} + \frac{2 u^{3} v \frac{d}{d u} \xi}{3 \varepsilon^{2}} - \frac{u^{2} \frac{d}{d u} \xi}{\varepsilon^{2}}&\nonumber\\
  - \frac{2 u v \frac{d}{d u} \xi}{\varepsilon^{2}} - \frac{v^{2} \frac{d}{d u} \xi}{\varepsilon^{2}}&=0\label{eq:BZ_lin_sym_1}
\end{align}
and the second linearised symmetry condition is given by
\begin{align}
\eta_1 - u^{2} \pdv{}{v} \xi - u \pdv{}{v} \eta_2 + u \pdv{\xi}{t} + \pdv{\eta_2}{t} - \frac{u^{4} \pdv{}{u} \xi}{3 \varepsilon} - \frac{u^{3} \pdv{}{u} \eta_2}{3 \varepsilon} + \frac{u^{2} \pdv{}{u} \xi}{\varepsilon} + \frac{u v \pdv{}{u} \xi}{\varepsilon} + \frac{u \pdv{}{u} \eta_2}{\varepsilon} + \frac{v \pdv{}{u} \eta_2}{\varepsilon}&=0.\label{eq:BZ_lin_sym_1}
\end{align}



\subsection{Ans\"atze assuming that the equations for the tangents are independent of the states}
The assumption that the derivatives of the tangents are at most functions of the time $t$ amounts to finding the roots of a polynomial of the states $u$ and $v$. Since these monomials are linearly independent we obtain the following equations:
\begin{align}
  1:&\pdv{\eta_1}{t} - \frac{1}{\varepsilon}\left(\eta_1 - \eta_2\right)=0,\\
  v:&\pdv{\eta_1}{u} - \pdv{\xi}{t}=0,\\
  v^{2}:&- \pdv{\xi}{u}=0,\\
  u:&-\pdv{\eta_1}{v} + \frac{1}{\varepsilon}\left(\pdv{\eta_1}{u}  - \pdv{\xi}{t}\right)=0,\\  
  u v:&\varepsilon\pdv{\xi}{v} - 2 \pdv{\xi}{u}=0,\\
  u^{2}:&\eta_1 + \pdv{\xi}{v} - \frac{1}{\varepsilon} \pdv{\xi}{u}=0,\\
  u^{3}:&- \pdv{\eta_1}{u} + \pdv{\xi}{t}=0,\\
u^{3} v:&\pdv{\xi}{u}=0,\\
 u^{4}:&- \frac{1}{3}\pdv{\xi}{v} + \frac{2}{3 \varepsilon}\pdv{\xi}{u}=0,\\
 u^{6}:&- \pdv{\xi}{u}=0.
\end{align}
for the first linearised symmetry condition and the following equations
\begin{align}
1:&\eta_1 + \pdv{\eta_2}{t}=0\\
v:&\pdv{\eta_2}{u}=0\\
u:&- \pdv{\eta_2}{v}  + \pdv{\xi}{t} + \frac{1}{\varepsilon}\pdv{\eta_2}{u} =0\\
u v:&\pdv{\xi}{u}=0\\
u^{2}:&- \pdv{\xi}{v} + \frac{1}{\varepsilon}\pdv{\xi}{u} =0\\
u^{3}:&- \pdv{\eta_2}{u}=0\\
u^{4}:&- \pdv{\xi}{u}=0
\end{align}
for the second linearised symmetry condition. 