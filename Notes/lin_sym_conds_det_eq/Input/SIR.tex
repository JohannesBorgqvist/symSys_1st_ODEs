We remind ourselves that we want to study the SIR model
\begin{equation*}
  \begin{split}
    \dv{S}{t}&=-rSI,\\
    \dv{I}{t}&=rSI-aI.\\
    \dv{R}{t}&=aI.\\    
    \end{split}
  \end{equation*}
  and we are looking for a generator of the following kind:
\begin{equation}
X=\xi(t,S,I,R)\partial_t+\eta_1(t,S,I,R)\partial_S+\eta_2(t,S,I,R)\partial_I+\eta_3(t,S,I,R)\partial_R.
\end{equation}
The first linearised symmetry condition is given by the following equation:
\begin{align}
  - a r I^{2} S \pdv{\xi}{I} + a r I^{2} S \pdv{\xi}{R} - a I \pdv{\eta_1}{I} + a I \pdv{\eta_1}{R}&\nonumber\\
  + r^{2} I^{2} S^{2} \pdv{\xi}{I} - r^{2} I^{2} S^{2} \pdv{\xi}{S} + r I S \pdv{\eta_1}{I} - r I S \pdv{\eta_1}{S}&\nonumber\\
  + r I S \pdv{\xi}{t} + r I\eta_1 + r S\eta_2 + \pdv{\eta_1}{t}&=0.\label{eq:SIR_lin_sym_1}
\end{align}
The second linearised symmetry condition is given by the following equation:
\begin{align}
  - a^{2} I^{2} \pdv{\xi}{I} + a^{2} I^{2} \pdv{\xi}{R} + 2 a r I^{2} S \pdv{\xi}{I} - a r I^{2} S \pdv{\xi}{R} - a r I^{2} S \pdv{\xi}{S}&\nonumber\\
  - a I \pdv{\eta_2}{I} + a I \pdv{\eta_2}{R} + a I \pdv{\xi}{t} + a\eta_2&\nonumber\\
  - r^{2} I^{2} S^{2} \pdv{\xi}{I} + r^{2} I^{2} S^{2} \pdv{\xi}{S} + r I S \pdv{\eta_2}{I} - r I S \pdv{\eta_2}{S}&\nonumber\\
  - r I S \pdv{\xi}{t} - r I\eta_1 - r S\eta_2 + \pdv{\eta_2}{t}&=0.\label{eq:SIR_lin_sym_2}
\end{align}
The third linearised symmetry condition is given by the following equation:
\begin{align}
  a^{2} I^{2} \pdv{\xi}{I} - a^{2} I^{2} \pdv{\xi}{R} - a r I^{2} S \pdv{\xi}{I} + a r I^{2} S \pdv{\xi}{S}&\nonumber\\
  - a I \pdv{\eta_3}{I} + a I \pdv{\eta_3}{R} - a I \pdv{\xi}{t} - a\eta_2&\nonumber\\
  + r I S \pdv{\eta_3}{I} - r I S \pdv{\eta_3}{S} + \pdv{\eta_3}{t}&=0.\label{eq:SIR_lin_sym_3}
\end{align}
\subsection{Ans\"atze assuming that the equations for the tangents are independent of the states}
The assumption that the derivatives of the tangents are at most functions of the time $t$ amounts to finding the roots of a polynomial of the states $u$ and $v$. Since these monomials are linearly independent we obtain the following equations:
\begin{align}
1:&\pdv{\eta_1}{t}=0,\\
I:&- a \pdv{\eta_1}{I} + a \pdv{\eta_1}{R} + r\eta_1=0,\\
S:&r\eta_2=0,\\
S I:&r \pdv{\eta_1}{I} - r \pdv{\eta_1}{S} + r \pdv{\xi}{t}=0,\\
S I^{2}:&- a r \pdv{\xi}{I} + a r \pdv{\xi}{R}=0,\\
S^{2} I^{2}:&r^{2} \pdv{\xi}{I} - r^{2} \pdv{\xi}{S}=0,
\end{align}
for the first linearised symmetry condition, the following equations
\begin{align}
1:&a\eta_2 + \pdv{\eta_2}{t}=0,\\
I:&- a \pdv{\eta_2}{I} + a \pdv{\eta_2}{R} + a \pdv{\xi}{t} - r\eta_1=0,\\
I^{2}:&- a^{2} \pdv{\xi}{I} + a^{2} \pdv{\xi}{R}=0,\\
S:&- r\eta_2=0,\\
S I:&r \pdv{\eta_2}{I} - r \pdv{\eta_2}{S} - r \pdv{\xi}{t}=0,\\
S I^{2}:&2 a r \pdv{\xi}{I} - a r \pdv{\xi}{R} - a r \pdv{\xi}{S}=0,\\
S^{2} I^{2}:&- r^{2} \pdv{\xi}{I} + r^{2} \pdv{\xi}{S}=0,
\end{align}
for the second linearised symmetry condition and the following equations
\begin{align}
1:&- a\eta_2 + \pdv{\eta_3}{t}=0,\\
I:&- a \pdv{\eta_3}{I} + a \pdv{\eta_3}{R} - a \pdv{\xi}{t}=0,\\
I^{2}:&a^{2} \pdv{\xi}{I} - a^{2} \pdv{\xi}{R}=0,\\
S I:&r \pdv{\eta_3}{I} - r \pdv{\eta_3}{S}=0,\\
S I^{2}:&- a r \pdv{\xi}{I} + a r \pdv{\xi}{S}=0,
\end{align}
for the third linearised symmetry condition.