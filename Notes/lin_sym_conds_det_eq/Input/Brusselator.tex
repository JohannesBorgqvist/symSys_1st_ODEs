We remind ourselves that we want to study the Brusselator model

\begin{equation*}
  \begin{split}
    \dv{u}{t}&=1-(b-1)u+au^2 v,\\
    \dv{v}{t}&=bu-au^2 v.\\
    \end{split}
\end{equation*}
and we are looking for a generator of the following kind:
\begin{equation}
X=\xi(t,u,v)\partial_t+\eta_1(t,u,v)\partial_u+\eta_2(t,u,v)\partial_v.
\end{equation}
Now, the first linearised symmetry condition is
\begin{align}
  - a^{2} u^{4} v^{2} \pdv{\xi}{u} + a^{2} u^{4} v^{2} \pdv{\xi}{v} + 2 a b u^{3} v \pdv{\xi}{u} - 2 a b u^{3} v \pdv{\xi}{v}&\nonumber\\
  - 2 a \eta_1 u v - a \eta_2 u^{2} - 2 a u^{3} v \pdv{\xi}{u} + a u^{3} v \pdv{\xi}{v}&\nonumber\\
  + a u^{2} v \pdv{\eta_1}{u} - a u^{2} v \pdv{\eta_1}{v} - 2 a u^{2} v \pdv{\xi}{u} + a u^{2} v \pdv{\xi}{v}&\nonumber\\
  - a u^{2} v \pdv{\xi}{t} - b^{2} u^{2} \pdv{\xi}{u} + b^{2} u^{2} \pdv{\xi}{v} + b \eta_1&\nonumber\\
  + 2 b u^{2} \pdv{\xi}{u} - b u^{2} \pdv{\xi}{v} - b u \pdv{\eta_1}{u} + b u \pdv{\eta_1}{v} &\nonumber\\
  + 2 b u \pdv{\xi}{u} - b u \pdv{\xi}{v} + b u \pdv{\xi}{t} - \eta_1&\nonumber\\
  - u^{2} \pdv{\xi}{u} + u \pdv{\eta_1}{u} - 2 u \pdv{\xi}{u} - u \pdv{\xi}{t} + \pdv{\eta_1}{u}&\nonumber\\
  - \pdv{\xi}{u} + \pdv{\eta_1}{t} - \pdv{\xi}{t}&=0,\label{eq:Brusselator_lin_sym_1}
\end{align}
while the second linearised symmetry condition is given by
\begin{align}
  a^{2} u^{4} v^{2} \pdv{\xi}{u} - a^{2} u^{4} v^{2} \pdv{\xi}{v} - 2 a b u^{3} v \pdv{\xi}{u} + 2 a b u^{3} v \pdv{\xi}{v}&\nonumber\\
  + 2 a \eta_1 u v + a \eta_2 u^{2} + a u^{3} v \pdv{\xi}{u} + a u^{2} v \pdv{\eta_2}{u}&\nonumber\\
  - a u^{2} v \pdv{\eta_2}{v} + a u^{2} v \pdv{\xi}{u} + a u^{2} v \pdv{\xi}{t} + b^{2} u^{2} \pdv{\xi}{u}&\nonumber\\
  - b^{2} u^{2} \pdv{\xi}{v} - b \eta_1 - b u^{2} \pdv{\xi}{u} - b u \pdv{\eta_2}{u}&\nonumber\\
  + b u \pdv{\eta_2}{v} - b u \pdv{\xi}{u} - b u \pdv{\xi}{t} + u \pdv{\eta_2}{u}&\nonumber\\
  + \pdv{\eta_2}{u} + \pdv{\eta_2}{t}&=0.\label{eq:Brusselator_lin_sym_2}
\end{align}


\subsection{Ans\"atze assuming that the equations for the tangents are independent of the states}
Again, the assumption that the derivatives of the tangents are at most functions of the time $t$ amounts to finding the roots of a polynomial of the states $u$ and $v$. Since these monomials are linearly independent we obtain the following equations:
\begin{align}
1:&b \eta_1 - \eta_1 + \pdv{\eta_1}{u} - \pdv{\xi}{u} + \pdv{\eta_1}{t} - \pdv{\xi}{t}=0\\
u:&- b \pdv{\eta_1}{u} + b \pdv{\eta_1}{v} + 2 b \pdv{\xi}{u} - b \pdv{\xi}{v} + b \pdv{\xi}{t} + \pdv{\eta_1}{u} - 2 \pdv{\xi}{u} - \pdv{\xi}{t}=0\\
u v:&- 2 a \eta_1=0\\
u^{2}:&- a \eta_2 - b^{2} \pdv{\xi}{u} + b^{2} \pdv{\xi}{v} + 2 b \pdv{\xi}{u} - b \pdv{\xi}{v} - \pdv{\xi}{u}=0\\
u^{2} v:&a \pdv{\eta_1}{u} - a \pdv{\eta_1}{v} - 2 a \pdv{\xi}{u} + a \pdv{\xi}{v} - a \pdv{\xi}{t}=0\\
u^{3} v:&2 a b \pdv{\xi}{u} - 2 a b \pdv{\xi}{v} - 2 a \pdv{\xi}{u} + a \pdv{\xi}{v}=0\\
u^{4} v^{2}:&- a^{2} \pdv{\xi}{u} + a^{2} \pdv{\xi}{v}=0,
\end{align}
for the first linearised symmetry condition and the following equations
\begin{align}
1:&- b \eta_1 + \pdv{\eta_2}{u} + \pdv{\eta_2}{t}=0\\
u:&- b \pdv{\eta_2}{u} + b \pdv{\eta_2}{v} - b \pdv{\xi}{u} - b \pdv{\xi}{t} + \pdv{\eta_2}{u}=0\\
u v:&2 a \eta_1=0\\
u^{2}:&a \eta_2 + b^{2} \pdv{\xi}{u} - b^{2} \pdv{\xi}{v} - b \pdv{\xi}{u}=0\\
u^{2} v:&a \pdv{\eta_2}{u} - a \pdv{\eta_2}{v} + a \pdv{\xi}{u} + a \pdv{\xi}{t}=0\\
u^{3} v:&- 2 a b \pdv{\xi}{u} + 2 a b \pdv{\xi}{v} + a \pdv{\xi}{u}=0\\
u^{4} v^{2}:&a^{2} \pdv{\xi}{u} - a^{2} \pdv{\xi}{v}=0,
\end{align}
for the second linearised symmetry condition.