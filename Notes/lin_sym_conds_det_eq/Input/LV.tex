We remind ourselves that we want to study the Lotka-Volterra model:

\begin{equation}
  \begin{split}
    \dv{u}{t}&=u(1-v),\\
    \dv{v}{t}&=\alpha v(u-1),\\    
    \end{split}
  \label{eq:LV}
\end{equation}
and we are looking for a generator of the following kind:
\begin{equation}
X=\xi(t,u,v)\partial_t+\eta_1(t,u,v)\partial_u+\eta_2(t,u,v)\partial_v.
\end{equation}
Now, derive the linearised symmetry conditions in equation \eqref{eq:lin_sym_practice} by plugging in our model in equation \eqref{eq:LV}. Given that we have autnomous reaction terms, our linearised symmetry conditions can be written as follows:


\begin{align}
    &\pdv{\eta_1}{t}+u(1-v)\pdv{\eta_1}{u}+\alpha v(u-1)\pdv{\eta_1}{v}\nonumber\\
    &-\left[u(1-v)\right]\left(\pdv{\xi}{t}+u(1-v)\pdv{\xi}{u}+\alpha v(u-1)\pdv{\xi}{v}\right)\label{eq:LV_lin_sym_1}\\
    &=\eta_1(1-v)-\eta_2 u,\nonumber\\
    &\pdv{\eta_2}{t}+u(1-v)\pdv{\eta_2}{u}+\alpha v(u-1)\pdv{\eta_2}{v}\nonumber\\
    &-\left[\alpha v(u-1)\right]\left(\pdv{\xi}{t}+u(1-v)\pdv{\xi}{u}+\alpha v(u-1)\pdv{\xi}{v}\right)\label{eq:LV_lin_sym_2}\\
    &=\alpha\eta_1 v+\alpha\eta_2(u-1).\nonumber
\end{align}




Now, let's expand this as much as possible and then try to derive the determining equations. Equation \eqref{eq:LV_lin_sym_1} is expanded as follows



\begin{align}
  \alpha u^{2} v^{2} \pdv{\xi}{v}  - \alpha u^{2} v \pdv{\xi}{v} - \alpha u v^{2} \pdv{\xi}{v} + \alpha u v \pdv{\eta_1}{v}&\nonumber\\
  + \alpha u v \pdv{\xi}{v}  - \alpha v \pdv{\eta_1}{v}  + \eta_1 v - \eta_1 + \eta_2 u &\nonumber\\
  - u^{2} v^{2} \pdv{\xi}{u}  + 2 u^{2} v \pdv{\xi}{u}  - u^{2} \pdv{\xi}{u}  - u v \pdv{\eta_1}{u}&\label{eq:LV_lin_sym_1_expanded}\\
  + u v \pdv{\xi}{t}+ u \pdv{\eta_1}{u}  - u \pdv{\xi}{t}+ \pdv{\eta_1}{t}&=0\nonumber
\end{align}
and here we see that we essentially have a polynomial where the different monomials are powers of the states $u$ and $v$. Now, since all these monomials are \textit{linearly independent} it follows that all coefficients in front of these monomials must be zero. This is what gives us our \textit{determining equations}, and more precisely here are our determining equations:



\begin{align}
1:&- \eta_1 +\pdv{\eta_1}{t}=0,\\
v:&- \alpha \pdv{\eta_1}{v}  + \eta_1=0,\\
u:&\eta_2 + \pdv{\eta_1}{u}  - \pdv{\xi}{t}=0,\\
u v:&\alpha \pdv{\eta_1}{v}  + \alpha \pdv{\xi}{v}  - \pdv{\eta_1}{u}  + \pdv{\xi}{t}=0,\\
u v^{2}:&- \alpha \pdv{\xi}{v} =0,\\
u^{2}:&- \pdv{\xi}{u} =0,\\
u^{2} v:&- \alpha \pdv{\xi}{v}  + 2 \pdv{\xi}{u} =0,\\
u^{2} v^{2}:&\alpha \pdv{\xi}{v} - \pdv{\xi}{u}=0.
\end{align}


Similarly, the second linearised symmetry condition in equation \eqref{eq:LV_lin_sym_2} is expanded as follows:



\begin{align}
  - \alpha^{2} u^{2} v^{2} \pdv{\xi}{v}  + 2 \alpha^{2} u v^{2} \pdv{\xi}{v} - \alpha^{2} v^{2} \pdv{\xi}{v} - \alpha \eta_1 v - \alpha \eta_2 u&\nonumber\\
  + \alpha \eta_2 + \alpha u^{2} v^{2} \pdv{\xi}{u} - \alpha u^{2} v \pdv{\xi}{u} - \alpha u v^{2} \pdv{\xi}{u} + \alpha u v \pdv{\eta_2}{v}&\nonumber\\
  + \alpha u v \pdv{\xi}{u}  - \alpha u v \pdv{\xi}{t} - \alpha v \pdv{\eta_2}{v}  + \alpha v \pdv{\xi}{t}&\label{eq:LV_lin_sym_1_expanded}\\
  - u v \pdv{\eta_2}{u}  + u \pdv{\eta_2}{u}  + \pdv{\eta_2}{t}&=0\nonumber
\end{align}
and again equation can be viewed as finding the roots of a polynomial which has monomials determined by the states $u$ and $v$. Since these monomials are linearly independent we can derive the determining equation stemming from this linearised symmetry condition:



\begin{align}
1:&\alpha \eta_2 + \pdv{\eta_2}{t}=0,\\
v:&- \alpha \eta_1 - \alpha \pdv{\eta_2}{v}  + \alpha \pdv{\xi}{t}=0,\\
v^{2}:&- \alpha^{2} \pdv{\xi}{v} =0,\\
u:&- \alpha \eta_2 + \pdv{\eta_2}{u} =0,\\
u v:&\alpha \pdv{\eta_2}{v}  + \alpha \pdv{\xi}{u}  - \alpha \pdv{\xi}{t} - \pdv{\eta_2}{u} =0,\\
u v^{2}:&2 \alpha^{2} \pdv{\xi}{v}  - \alpha \pdv{\xi}{u} =0,\\
u^{2} v:&- \alpha \pdv{\xi}{u} =0,\\
u^{2} v^{2}:&- \alpha^{2} \pdv{\xi}{v}  + \alpha \pdv{\xi}{u} =0.
\end{align}