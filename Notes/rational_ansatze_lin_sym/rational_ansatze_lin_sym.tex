%--------------------------------------------------------------------------------------
% Definition of the document
%--------------------------------------------------------------------------------------
\documentclass[12pt]{article}
%--------------------------------------------------------------------------------------
% Define packages needed for the writing
%--------------------------------------------------------------------------------------
% General document formatting
\usepackage[margin=1in]{geometry}
%\usepackage[parfill]{parskip}
\usepackage[utf8]{inputenc}    
% Related to math
\usepackage{amsmath,amssymb,amsfonts,amsthm}
% To write derivative more efficiently
\usepackage{physics}
% To add figures
\usepackage{graphicx}
%--------------------------------------------------------------------------------------
% Define the specific setup for the document
% --------------------------------------------------------------------------------------
\newtheorem{theorem}{Theorem}[section]
\theoremstyle{definition}
\newtheorem{definition}{Definition}[section]
\newtheorem{corollary}{Corollary}[theorem]
\theoremstyle{remark}
\newtheorem*{remark}{Remark}
%--------------------------------------------------------------------------------------
% Define the specific setup for the document
%--------------------------------------------------------------------------------------
% Set the font of the equations to Helvetica as well. 
% Increase the spacing between the lines
%\linespread{1.5}
%--------------------------------------------------------------------------------------
% THE DOCUMENTS BEGINS
%--------------------------------------------------------------------------------------
\begin{document}
\title{\textbf{Plugging in rational ans\"atze into the linearised symmetry condition for the phase plane}}
\author{Johannes Borgqvist}
\date{\today}
\maketitle
\tableofcontents
\clearpage
%--------------------------------------------------------------------------------------
% THE LV-model
\section{LV-model}
The LV-model in the phase plane is given by:
\begin{equation}
\dv{v}{u}=\dfrac{a v \left(u - 1\right)}{u \left(1 - v\right)}
\label{eq:LV}
\end{equation}
The monomials are:
\begin{equation}
M=\left[\begin{matrix}1\\v\\v^{2}\\u\\u v\\u^{2}\end{matrix}\right].
\end{equation}

The unknown coefficient in our polynomial ansatze are:
\begin{equation}
\mathbf{c}=\left[\begin{matrix}c_{0}\\c_{1}\\c_{2}\\c_{3}\\c_{4}\\c_{5}\\c_{6}\\c_{7}\\c_{8}\\c_{9}\\c_{10}\\c_{11}\\c_{12}\\c_{13}\\c_{14}\\c_{15}\\c_{16}\\c_{17}\\c_{18}\\c_{19}\\c_{20}\\c_{21}\\c_{22}\\c_{23}\end{matrix}\right].
\end{equation}

Now, we use rational ansatze of the type $\eta_1=P_1/P_2$ and $\eta_2=P_3/P_4$. Our four polynomials are:

\begin{align*}
P_1&=c_{0} + c_{1} v + c_{2} v^{2} + c_{3} u + c_{4} u v + c_{5} u^{2},\\
P_2&=c_{10} u v + c_{11} u^{2} + c_{6} + c_{7} v + c_{8} v^{2} + c_{9} u,\\
P_3&=c_{12} + c_{13} v + c_{14} v^{2} + c_{15} u + c_{16} u v + c_{17} u^{2},\\
P_4&=c_{18} + c_{19} v + c_{20} v^{2} + c_{21} u + c_{22} u v + c_{23} u^{2}.\\
\end{align*}
The system of equations resulting from plugging in these ansatze into the linearised symmetry condition contains 73 equations.

%--------------------------------------------------------------------------------------
% THE DOCUMENT ENDS
%--------------------------------------------------------------------------------------
\end{document}