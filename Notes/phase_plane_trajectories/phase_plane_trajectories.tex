%--------------------------------------------------------------------------------------
% Definition of the document
%--------------------------------------------------------------------------------------
\documentclass[12pt]{article}
%--------------------------------------------------------------------------------------
% Define packages needed for the writing
%--------------------------------------------------------------------------------------
% General document formatting
\usepackage[margin=1in]{geometry}
%\usepackage[parfill]{parskip}
\usepackage[utf8]{inputenc}    
% Related to math
\usepackage{amsmath,amssymb,amsfonts,amsthm}
% To write derivative more efficiently
\usepackage{physics}
% To add figures
\usepackage{graphicx}
%--------------------------------------------------------------------------------------
% Define the specific setup for the document
% --------------------------------------------------------------------------------------
\newtheorem{theorem}{Theorem}[section]
\theoremstyle{definition}
\newtheorem{definition}{Definition}[section]
\newtheorem{corollary}{Corollary}[theorem]
\theoremstyle{remark}
\newtheorem*{remark}{Remark}
%--------------------------------------------------------------------------------------
% Define the specific setup for the document
%--------------------------------------------------------------------------------------
% Set the font of the equations to Helvetica as well. 
% Increase the spacing between the lines
\linespread{1.5}
% Set the graphics-path, i.e. where we store the figures
\graphicspath{{./images/}}
%--------------------------------------------------------------------------------------
% THE DOCUMENTS BEGINS
%--------------------------------------------------------------------------------------
\begin{document}
\title{\textbf{Fibre-preserving symmetries of time-invariant models in mathematical biology acting on the phase plane}}
\author{Johannes Borgqvist}
\date{\today}
\maketitle
\tableofcontents
\clearpage
%--------------------------------------------------------------------------------------
% THE INTRO
\section{Introduction}


Okay, so we got rejected from the bulletin of Mathematical Biology. Overall, the two reviewers thought that the researchs aim and research questions were brilliant and they thought that our manuscript was very well-written. However, all the content was trivial and there was nothing new there. So they gave us three months to re-submit a new manuscript. The initial plan was to go through a bunch of famous models in mathematical biology and then present their symmetries as well as the differential invariants and so on. The thing that stopped us from doing this was the fact that our symbolic solver could not find the symmetries of these models.

So my suggestion here is that we try to find the symmetries of these models by hand essentially. Again, we study the following type of system of first order ODEs

\begin{equation}
\dv{y_i}{t} = \omega_i(t,y_1,\ldots,y_k), \quad i=1,\ldots,k,
  \label{eq:sys_ODE}
\end{equation}
where the infinitesimal generator of the Lie group is given by
\begin{equation}
  X=\xi\partial_t+\eta_{1}\partial{y_1}+\ldots+\eta_{k}\partial{y_k}
  \label{eq:generator}
\end{equation}
and the prolonged generator is given by
\begin{equation}
  X^{(1)}=X+\eta_{1}^{(1)}\partial{y_1}+\ldots+\eta_{k}^{(1)}\partial{y_k}.
  \label{eq:prolonged}
\end{equation}
Now, given this prolonged generator, the \textit{linearised symmetry conditions} are defined as follows:
\begin{equation}
X^{(1)} \left( \dv{y_i}{t} - \omega_i(t,y_1,\ldots,y_k) \right) = 0 , \quad i=1,\ldots,k.
\end{equation}
In particular, using the \textit{total derivative}
\begin{equation}
D_t=\partial_t+y_1'\partial{y_1}+\ldots+y_k'\partial{y_k}
  \label{eq:tot_der}
  \end{equation}
these symmetry conditions can be written as follows
\begin{equation}
\label{eq:lin_sym_practice}
D_t\eta_i-\omega_i D_t\xi=X\left(\omega_i(t,y_1,\ldots,y_k)\right),\quad i=1,\ldots,k \,.
\end{equation}
So to increase the impact, we probably need to solve equation \eqref{eq:lin_sym_practice} for a bunch of biologically relevant models. This document is the start of that journey, and below I will list the models that I figured that we can focus on.

The Lotka-Volterra model:

\begin{equation}
  \begin{split}
    \dv{u}{t}&=u(1-v),\\
    \dv{v}{t}&=\alpha v(u-1).\\    
    \end{split}
  \label{eq:LV}
\end{equation}

The BZ model

\begin{equation}
  \begin{split}
    \dv{u}{t}&=\dfrac{1}{\varepsilon}v-\dfrac{1}{\varepsilon}\left(\dfrac{1}{3}u^3-u\right),\\
    \dv{v}{t}&=-u.\\    
    \end{split}
  \label{eq:BZ}
\end{equation}

The Lorenz equations:

\begin{equation}
  \begin{split}
    \dv{u}{t}&=a(v-u),\\
    \dv{v}{t}&=-uw+bu-v,\\
    \dv{w}{t}&=uv-cw.
    \end{split}
  \label{eq:Lorenz}
\end{equation}

The Brusselator:

\begin{equation}
  \begin{split}
    \dv{u}{t}&=1-(b-1)u+au^2 v,\\
    \dv{v}{t}&=bu-au^2 v.\\
    \end{split}
  \label{eq:Brusselator}
\end{equation}

The SIR model:
\begin{equation}
  \begin{split}
    \dv{S}{t}&=-rSI,\\
    \dv{I}{t}&=rSI-aI.\\
    \dv{R}{t}&=aI.\\    
    \end{split}
  \label{eq:SIR}
\end{equation}
The MM system:
\begin{equation}
  \begin{split}
    \dv{s}{t}&=-k_1 es+k_{-1}c,\\
    \dv{c}{t}&=k_1 es-(k_{-1}+k_2)c,\\
    \dv{e}{t}&=-k_1 es+(k_{-1}+k_2)c,\\
    \dv{p}{t}&=k_2 c.\\        
    \end{split}
  \label{eq:MM}
\end{equation}
The Goodwin model (with n=1):
\begin{equation}
\begin{split}
  \dv{R}{t}&=-b_1 R+\dfrac{K}{1+\beta T^n}=\omega_1 (R,L,T)\\
  \dv{L}{t}&=g_1 R- b_2 L=\omega_2 (R,L,T)\\
    \dv{T}{t}&=g_2 L-b_3 T=\omega_3 (R,L,T)
\end{split}
  \label{eq:Goodwin}
\end{equation}

So let's go through these models one by one and see if we can find any symmetries. Let's start with the Lotka-Volterra model!
%--------------------------------------------------------------------------------------
%--------------------------------------------------------------------------------------
% Fibre preserving symmetries
\section{Mathematical theory of symmetries of differential equations}
Here, we will present a condensed version of the mathematical theory of symmetries of differential equations. For the interested reader, there are many excellent introductory texts~\cite{bluman1989symmetries,hydon2000symmetry,olver2000applications,olver2008equivalence,lang2001,nakahara2003} and here we will focus on general two component of first order ODEs

\begin{equation}
  \begin{split}
    \dv{u}{\tau}&=\omega_1(t,u,v),\\
    \dv{v}{\tau}&=\omega_2(t,u,v),\\    
    \end{split}
  \label{eq:sys_general}
\end{equation}
describing the time evolution of either the concentration profile of two proteins $u(\tau)$ and $v(\tau)$ or the evolution of two populations. Given this system, we subsequently will present the notions of Lie-transformations, their prolongations and symmetries of differential equations. 


% Lie groups of infinitesimal transformations
\subsection{Lie groups of infinitesimal transformations}
Here, a one parameter Lie-symmetry meaning a one parameter $\mathcal{C}^{\infty}$-diffeomorphism is a transformation
\begin{equation}
  (\hat{\tau}(t,u(\tau),v(\tau);\epsilon),\hat{u}(t,u(\tau),v(\tau);\epsilon),\hat{v}(t,u(\tau),v(\tau);\epsilon))=\left(\phi_1(\tau,u,v;\epsilon),\phi_2(\tau,u,v;\epsilon),\phi_3(\tau,u,v;\epsilon)\right)
  \label{eq:symmetry_original}
  \end{equation}
parameterised by the parameter $\epsilon$ where the transformation is defined by the infinitely differentiable functions $\phi_i\in\mathcal{C}^{\infty}(\mathbb{R}^4),\quad i=1,2,3$. Now, we will parametrise such symmetries in terms of $\epsilon$ in the following way
\begin{equation}
\Gamma_{\epsilon}:(\tau,u(\tau),v(\tau))\mapsto\left(\hat{\tau}(\epsilon),\hat{u}(\epsilon),\hat{v}(\epsilon)\right)
\label{eq:symmetry_general}
\end{equation}
and the characterising feature of a symmetry of a differential equation is that if $(\tau,u(\tau),v(\tau))$ is a solution to the system of ODEs in Equation \eqref{eq:sys_general} then so is $(\hat{\tau}(\epsilon),\hat{u}(\epsilon),\hat{v}(\epsilon))$. The set of such one parameter Lie-transformations $G$ together with a multiplication operation $\times$, forms a \textit{one parameter Lie group of transformations} denoted by $(G,\times)$. Such a Lie-group has three defining properties:

\begin{enumerate}
    \item \textit{Multiplication}: For two transformation parameters $\epsilon,\delta\in\mathbb{R}$, multiplication of symmetries (meaning that we first transform with $\delta$ and then with $\epsilon$) is defined by: $\Gamma_\epsilon\times\Gamma_\delta=\Gamma_{\epsilon+\delta}$,
    \item \textit{Identity element}: The trivial symmetry $\Gamma_0=\Gamma_{\epsilon=0}$ acts trivially on curves: $\Gamma_0 (\tau,u(\tau),v(\tau))=(\tau,u(\tau),v(\tau))$,
    \item \textit{Inverse element}: The inverse symmetry $\Gamma^{-1}_{\epsilon}$ is defined by $\Gamma^{-1}_{\epsilon}=\Gamma_{-\epsilon}$.
\end{enumerate}
Moreover, by the continuity of these transformations we can Taylor expand $\Gamma_{\epsilon}$ around $\epsilon\approx 0$ which gives us
\begin{align}
  \hat{\tau}&=\tau+\xi(\tau,u,v)\epsilon+\mathcal{O}(\epsilon^2),\\
  \hat{u}&=u+\eta_1(\tau,u,v)\epsilon+\mathcal{O}(\epsilon^2),\\
  \hat{v}&=v+\eta_2(\tau,u,v)\epsilon+\mathcal{O}(\epsilon^2),
\end{align}
and here the tangents $\xi$, $\eta_1$ and $\eta_2$ are referred to as the \textit{infinitesimals} which, in turn, are defined as follows:
\begin{align}
  \xi(\tau,u,v)&=\left.\pdv{\phi_1(\tau,u,v;\epsilon)}{\epsilon}\right|_{\epsilon=0},\label{eq:tangent_1}\\
  \eta_1(\tau,u,v)&=\left.\pdv{\phi_2(\tau,u,v;\epsilon)}{\epsilon}\right|_{\epsilon=0},\label{eq:tangent_2}\\
  \eta_2(\tau,u,v)&=\left.\pdv{\phi_3(\tau,u,v;\epsilon)}{\epsilon}\right|_{\epsilon=0},\label{eq:tangent_3}
\end{align}
where the functions $\phi_1,\phi_2,\phi_3\in\mathcal{C}^{\infty}(\mathbb{R}^{4})$ define the symmetry according to Equation \eqref{eq:symmetry_original}. Now, one of the most powerful results from the theory of Lie-symmetries is that the global behaviour of the symmetry in Equation \eqref{eq:symmetry_general} can be retrieved from the local behaviour in terms of the infinitesimals in Equations \eqref{eq:tangent_1} to \eqref{eq:tangent_3}. More precisely, the vector field defined by
\begin{equation}
X=\xi(\tau,u,v)\partial_t+\eta_1(\tau,u,v)\partial_u+\eta_2(\tau,u,v)\partial_v
  \label{eq:generator}
\end{equation}
is referred to as the \textit{infinitesimal generator of the Lie group}. Using this vector field, \textit{Lie's first fundamental theorem} \cite{bluman1989symmetries} says that the symmetry in Equation \eqref{eq:symmetry_general}, is in fact given by

\begin{equation}
\Gamma_{\epsilon}:(\tau,u,v)\mapsto \left(e^{\epsilon X}\tau,e^{\epsilon X}u,e^{\epsilon X}v\right)
  \label{eq:symmetry_Lie}
\end{equation}
where the exponential map is defined as follows
\begin{equation}
  e^{\epsilon X}=\sum_{j=0}^{\infty}\dfrac{\epsilon^j}{j!}X^{j}.
  \label{eq:exponential}
\end{equation}
Accordingly, it is enough to know the local behaviour represented by the infinitesimal generator of the Lie group $X$ in Equation \eqref{eq:generator} in order to also know the global behaviour of the symmetry according to Equation \eqref{eq:symmetry_Lie}. Thus, to find the symmetries we must find the infinitesimals which is possible owing to the fact that the defining property of symmetries can be expressed in terms of their local action. Before, we are able to present the notion of a symmetry of a differential equation, we must introduce the idea of so called extended transformations or prolongations.






% Prolongations: extended transformations
\subsection{Prolongations: extended transformations}
Our aim is to mathematically define a symmetry of a differential equation as a Lie-transformation that maps a solution curve of the differential equation to another solution curve. More precisely, given the Lie transformation
$$\Gamma_{\epsilon}:(\tau,u(\tau),v(\tau))\mapsto(\hat{t}(\epsilon),\hat{u}(\epsilon),\hat{v}(\epsilon))$$
we would say that $\Gamma_\epsilon$ is a symmetry of the system of differential equations in Equation \eqref{eq:sys_general} if it maps a solutions curve $(\tau,u(\tau),v(\tau))$ of this system of ODEs to another solution curve $(\hat{t}(\epsilon),\hat{u}(\epsilon),\hat{v}(\epsilon))$. Here, $\tau\in\mathcal{B}\sim\mathbb{R}$ is called the \textit{independent variable} and it defines the so called \textit{base space} $\mathcal{B}$ of the symmetry $\Gamma_\epsilon$. Also, the states $(u(\tau),v(\tau))\in F\sim\mathbb{R}^2$ are called the \textit{dependent variables} and they define the so called \textit{fibre} $F$ of the symmetry $\Gamma_\epsilon$. Given these spaces, the symmetry $\Gamma_\epsilon$ acts on the so called \textit{total space} $E$ defined as $E=B\times F\sim\mathbb{R}^3$, and thus we have that $\Gamma_\epsilon:E\mapsto E$. Now, a transformation acting on solutions to differential equations must account for the derivatives of the states, e.g. $u'(\tau)$ and $v'(\tau)$, and to this end we introduce the notion of extended transforamtions.

There is a natural extension of a Lie-symmetry $\Gamma_\epsilon$ referred to as the prolongation of the symmetry and it is defined by the derivatives of the states. More precisely, \textit{the first prolongation of the Lie-symmetry} $\Gamma^{(1)}_{\epsilon}$ is defined as follows
\begin{equation}
\Gamma_{\epsilon}^{(1)}:(\tau,u(\tau),v(\tau),u'(\tau),v'(\tau))\mapsto(\hat{t}(\epsilon),\hat{u}(\epsilon),\hat{v}(\epsilon),\hat{u}'(\epsilon),\hat{v}'(\epsilon))
  \label{eq:prolongation}
\end{equation}
where the derivatives of the states are defined by $u'(\tau)=\omega_1(t,u,v)$ and $v'(\tau)=\omega_1(t,u,v)$ according to Equation \eqref{eq:sys_general}. Here, it is not entirely clear how the derivatives $\hat{u}'(\epsilon)$ and $\hat{v}'(\epsilon)$ are defined, and to this end we need to introduce the notion of the \textit{total derivative} $D_\tau$. This operator is defined as follows

\begin{equation}
D_\tau=\partial_\tau+u'(\tau)\partial_u+v'(\tau)\partial_v
  \label{eq:tot_der}
  \end{equation}
  and given the total derivative the derivatives of the transformed coordinates are defined as follows
  \begin{equation}
    \begin{split}
      \hat{u}'(\epsilon)&=\dfrac{D_\tau \hat{u}(\tau,u,v;\epsilon)}{D_\tau\hat{\tau}(\tau,u,v;\epsilon)},\\
      \hat{v}'(\epsilon)&=\dfrac{D_\tau \hat{v}(\tau,u,v;\epsilon)}{D_\tau\hat{\tau}(\tau,u,v;\epsilon)}.\\
    \end{split}
    \label{eq:extended_derivatives}
    \end{equation}
    Moreover, the derivatives $(u'(\tau),v'(\tau))\in F'\sim\mathbb{R}^2$ define the \textit{prolonged fibre} $F'$, and \textit{the first jet space} $\mathcal{J}^{(1)}$ is defined by $\mathcal{J}^{(1)}=E\times F'$. Given the jet space, the prolonged symmetry can be succintly written in the following way $\Gamma^{(1)}_{\epsilon}:\mathcal{J}^{(1)}\mapsto\mathcal{J}^{(1)}$. Also, the operator $\Gamma_\epsilon\mapsto\Gamma^{(1)}_{\epsilon}$ is well-defined and is referred to as the \textit{lift} of the symmetry $\Gamma_{\epsilon}$. Previously, we showed that the infinitesimal action of the symmetry $\Gamma_{\epsilon}$ is expressed by the infinitesimal generator of the Lie group $X$ in Equation \eqref{eq:generator}, and similarly there is an infinitesimal representation of the prolonged symmetry $\Gamma^{(1)}_{\epsilon}$.

    Locally, we can describe the action of the first prolongation of the symmetry $\Gamma^{(1)}_{\epsilon}$ by the \textit{first prolongation of the infinitesimal generator of the Lie group} $X^{(1)}$. This operator is defined as follows
\begin{equation}
X^{(1)}=X+\eta_1^{(1)}(\tau,u,v)\partial_{u'}+\eta_2^{(1)}(\tau,u,v)\partial_{v'}
\label{eq:prolonged_generator}
\end{equation}
and here the prolonged infinitesimals $\eta_1^{(1)}$ and $\eta_{2}^{(1)}$ respectively are given by the \textit{prolongation formula}
\begin{equation}
\eta_{i}^{(1)}(\tau,u,v)=D_\tau\eta_i(\tau,u,v)-\omega_i(\tau,u,v) D_\tau\xi(\tau,u,v),\quad i=1,2.
  \label{eq:prolongation_formula}
\end{equation}
Now, given the notion of prolongations, we are now able to formulate the conditions that define a symmetry of a differential equation. 


% Lie symmetries of differential 
\subsection{Symmetries of differential equations}
Consider a one parameter Lie transformation $\Gamma_\epsilon:(\tau,u(\tau),v(\tau))\mapsto(\hat{\tau}(\epsilon),\hat{u}(\epsilon),\hat{v}(\epsilon))$. Then, this transformation is a symmetry of the system of differential equations in Equation \eqref{eq:sys_general} if it maps a solution curve $(\tau,u(\tau),v(\tau))$ to another solution curve $(\hat{\tau}(\epsilon),\hat{u}(\epsilon),\hat{v}(\epsilon))$. Using the notions of jet spaces and prolongations, it can be shown that a Lie-transformation $\Gamma_\epsilon$ is a symmetry of the system of differential equations in Equation \eqref{eq:sys_general} if and only if the following so called \textit{symmetry conditions} hold
\begin{equation}
 \begin{split}
   u'(\epsilon)&=\omega_1(\hat{\tau}(\epsilon),\hat{u}(\epsilon),\hat{v}(\epsilon))\quad\mathrm{whenever}\quad\dv{u}{\tau}=\omega_1(\tau,u,v),\\
   v'(\epsilon)&=\omega_2(\hat{\tau}(\epsilon),\hat{u}(\epsilon),\hat{v}(\epsilon))\quad\mathrm{whenever}\quad\dv{v}{\tau}=\omega_2(\tau,u,v),\\   
 \end{split}
 \label{eq:sym_cond}
\end{equation}
where the derivatives $u'(\epsilon)$ and $v'(\epsilon)$ are defined by Equation \eqref{eq:extended_derivatives}. In general, it is difficult to use these symmetry conditions and instead one can formulate the same condition in terms of the infinitesimal action of the prolonged symmetry. 



 More precsely, a Lie transformation $\Gamma_\epsilon$ is a symmetry of the system of differential equations in Equation \eqref{eq:sys_general} if and only if the \textit{linearised symmetry conditions} given by
\begin{equation}
  \begin{split}
    X^{(1)}\left(\dv{u}{\tau}-\omega_1(\tau,u,v)\right)&=0\quad\mathrm{whenever}\quad\dv{u}{\tau}=\omega_1(\tau,u,v),\\
    X^{(1)}\left(\dv{v}{\tau}-\omega_2(\tau,u,v)\right)&=0\quad\mathrm{whenever}\quad\dv{v}{\tau}=\omega_2(\tau,u,v),
    \end{split}
  \label{eq:lin_sym_ori}
  \end{equation}
are satisfied. By the linearity of the prolonged generator $X^{(1)}$ in Equation \eqref{eq:prolonged_generator}, these equations can in fact be written as follows \cite{stephani1989differential}:
\begin{equation}
  \begin{split}
    \eta_1^{(1)}(\tau,u,v)&=X(\omega_1(\tau,u,v))\quad\mathrm{whenever}\quad\dv{u}{\tau}=\omega_1(\tau,u,v),\\
    \eta_2^{(1)}(\tau,u,v)&=X(\omega_2(\tau,u,v))\quad\mathrm{whenever}\quad\dv{v}{\tau}=\omega_2(\tau,u,v),
    \end{split}
  \label{eq:lin_sym}
\end{equation}
where the prolonged tangents in the left hand sides are given by the prolongation formula in Equation \eqref{eq:prolongation_formula}. In general, the symmetries of a given differential equation are found by solving the linearised symmetry conditions for the infinitesimals and then the symmetry is retrieved using the exponential map. Next, we will focus on a common class of ODEs in mathematical biology, namely that of autonomous models and specifically time-invariant models. 

% Invariants and canonical coordinates
\subsection{Invariants and canonical coordinates}
In practice, the meaning of symmetries is often interpreted in terms of their invariants. Given the first prolongation of an infinitesimal generator of the Lie group
$$X^{(1)}=\xi(\tau,u,v)\partial_\tau+\eta_1(\tau,u,v)\partial_u+\eta_2(\tau,u,v)\partial_v+\eta_{1}^{(1)}(\tau,u,v)\partial_{u'}+\eta_{2}^{(1)}(\tau,u,v)\partial_{v}$$
an \textit{invariant} of this generator is a non-constant function $I=I(t,u,v,u',v')$ satisfying the following equation
\begin{equation}
  X^{(1)}\left(I\right)=0.
  \label{eq:invariant}
\end{equation}
In the light of this definition, we see that the linearised symmetry conditions in Equation \eqref{eq:lin_sym_ori} in fact corresponds to saying that the solution manifold of the system of ODEs itself is invariant under the infinitesimal action of the symmetry. In terms of interpretations, the invariants corresponds to conserved properties and in particular we can classify the invariants into two types. If the symmetry is trivial, meaning that it maps the points on a solution curve to other points on the same solution curve, then the invariants of this symmetry correspond to conservation laws, such as energy conservation, of the observed system. If the symmetry is non-trivial, meaning that the symmetry maps points on a solution curve to points on another distinct solution curve, then the invariant corresponds to properties that are conserved for numerous distinct solution curves. It is also possible to conduct a coordinate change where most of the transformed coordinates are invariants of the symmetry at hand, and such coordinates are referred to as canonical coordinates.

In practice, a symmetry $\Gamma_\epsilon$ is often calculated from its infinitesimal generator of the Lie group $X$ using so called \textit{canonical coordinates} rather than the exponential map in Equation \eqref{eq:symmetry_Lie}. Given an infinitesimal generator of the Lie group $X=\xi\partial_\tau+\eta_1\partial_u+\eta_2\partial_v$ expressed in the original coordinates $(\tau,u(\tau),v(\tau))$ of the total space, there exist another set of coordinates $(s,r_1,r_2)=(s(\tau,u(\tau),v(\tau)),r_1(\tau,u(\tau),v(\tau)),r_2(\tau,u(\tau),v(\tau)))$ such that the infinitesimal generator gets transformed into a translation generator in $s$, i.e. $X=\partial_s$. In other words, we want the following equation to hold
\begin{equation}
  \left.\dv{\hat{r}_1}{\epsilon}\right|_{\epsilon=0}=  \left.\dv{\hat{r}_2}{\epsilon}\right|_{\epsilon=0}=0,\quad  \left.\dv{\hat{s}}{\epsilon}\right|_{\epsilon=0}=1.
  \label{eq:canonical_1}
  \end{equation}
  or to express it in terms of the resulting symmetry
\begin{equation}
  \Gamma_{\epsilon}:(s,r_1,r_2)\mapsto(\hat{s}(\epsilon),\hat{r}_1(\epsilon),\hat{r}_2(\epsilon)),\quad(\hat{s},\hat{r}_1,\hat{r}_2)=(s(\hat{\tau},\hat{u},\hat{v}),r_1(\hat{\tau},\hat{u},\hat{v}),r_2(\hat{\tau},\hat{u},\hat{v}))=(s+\epsilon,r_1,r_2)
  \label{eq:canonical_2}
  \end{equation}  
  Using the chain rule on the transformed coordinates in Equation \eqref{eq:canonical_2}, we obtain
  \begin{align*}
    \dv{\hat{r}_1}{\epsilon}&=\dv{\hat{\tau}}{\epsilon}\dv{r_1}{\hat{\tau}}+\dv{\hat{u}}{\epsilon}\dv{r_1}{\hat{u}}+\dv{\hat{v}}{\epsilon}\dv{r_1}{\hat{v}},\\
    \dv{\hat{r}_2}{\epsilon}&=\dv{\hat{\tau}}{\epsilon}\dv{r_2}{\hat{\tau}}+\dv{\hat{u}}{\epsilon}\dv{r_1}{\hat{u}}+\dv{\hat{v}}{\epsilon}\dv{r_2}{\hat{v}},\\
    \dv{\hat{s}}{\epsilon}&=\dv{\hat{\tau}}{\epsilon}\dv{s}{\hat{\tau}}+\dv{\hat{u}}{\epsilon}\dv{s}{\hat{u}}+\dv{\hat{v}}{\epsilon}\dv{s}{\hat{v}},\\
  \end{align*}
  and next we can evaluate the above equations at $\epsilon=0$. The left hand sides are then given by Equation \eqref{eq:canonical_1}. Also, we have that $\hat{\tau}(\epsilon=0)=\tau$, $\hat{u}(\epsilon=0)=u$ and $\hat{v}(\epsilon=0)=v$. Lastly, the infinitesimals are defined as $\xi(\tau,u,v)=\left.\dd\hat{\tau}/\dd\epsilon\right|_{\epsilon=0}$, $\eta_1(\tau,u,v)=\left.\dd\hat{u}/\dd\epsilon\right|_{\epsilon=0}$ and $\eta_2(\tau,u,v)=\left.\dd\hat{v}/\dd\epsilon\right|_{\epsilon=0}$. All in all, this implies that the canonical coordinates $(s,r_1,r_2)$ satisfy the following equations

  \begin{equation}
    \begin{split}
     \xi(\tau,u,v)\dv{r_1}{\tau}+\eta_1(\tau,u,v)\dv{r_1}{u}+\eta_2(\tau,u,v)\dv{r_1}{v}&=0,\\
    \xi(\tau,u,v)\dv{r_2}{\tau}+\eta_1(\tau,u,v)\dv{r_1}{u}+\eta_2(\tau,u,v)\dv{r_2}{v}&=0,\\
    \xi(\tau,u,v)\dv{s}{\tau}+\eta_1(\tau,u,v)\dv{s}{u}+\eta_2(\tau,u,v)\dv{s}{v}&=1.\\      
  \end{split}
  \label{eq:canonical_final}
\end{equation}
Now, we can obtain the symmetry $\Gamma_\epsilon$ generated by an infinitesimal generator of the Lie group $X$ through its canonical coordinates. Firstly, we would calculate the canonical coordinates by solving the system in Equation \eqref{eq:canonical_final}. Then, since the symmetry is easily expressed in terms of its canonical coordinates as $\hat{s}(\tau,u,v;\epsilon)=s(\tau,u,v)+\epsilon$, $\hat{r}_1(\tau,u,v;\epsilon)=r_1(\tau,u,v)$ and $\hat{r}_2(\tau,u,v;\epsilon)=r_2(\tau,u,v)$, we can then convert back to the original coordinates to obtain the expression for the symmetry. 
% --------------------------------------------------------------------------------------
%--------------------------------------------------------------------------------------
% Fibre preserving symmetries
\section{Fibre-preserving symmetries of time-invariant models acting on the phase-plane}
A common class of models in Mathematical Biology is that of two state time-invariant models. For example, these models typically describe the evolution of two competing populations or the evolution of two reacting proteins. To this end, we will now consider the following autonomous (specifically time-invariant) two state system of first order ODEs
\begin{equation}
  \begin{split}
    \dv{u}{\tau}&=\omega_1(u,v),\\
    \dv{v}{\tau}&=\omega_2(u,v).
  \end{split}
  \label{eq:sys_auto}
\end{equation}
where the right hand sides $\omega_1(u,v)$ and $\omega_2(u,v)$ referred to as the \textit{reaction terms} are independent of the time $\tau$. For these types of systems, there are two well-known symmetries. The first one is the time translation symmetry 

\begin{equation}
  \begin{split}
    \Gamma_{\epsilon}&:(\tau,u,v)\mapsto(\tau+\epsilon,u,v)\\
   \textrm{generated by}&\\
    X&=\partial_\tau
  \end{split}
  \label{eq:time_trans}
\end{equation}
and this symmetry is common to all autonomous models. The second symmetry is the \textit{trivial symmetry} generated by the infinitesimal generator of the Lie group given by
\begin{equation}
X=\partial_\tau+\omega_1(u,v)\partial_u+\omega_2(u,v)\partial_v
  \label{eq:trivial}
\end{equation}
and this vector field is parallel to the vector field of the original system of ODEs in Equation \eqref{eq:sys_auto}. Consequently, this symmetry maps points on a solution curve onto other points on the \textit{same solution curve}. Note that the infinitesimals in the time direction $\xi$ of these two generators are \textit{independent of the states} $u$ and $v$ meaning that these generators do not mix the time and state dependence. Such symmetries are called \textit{fibre preserving}, and next we are interested in such symmetries that are restricted to the $(u,v)$-phase plane.\\

\hrule
\begin{definition}[Fibre preserving symmetries of time-invariant models restricted to the phase plane]
Consider the autonomous system of ODEs in Equation \eqref{eq:sys_auto}. A \textit{fibre-preserving symmetry} $\Gamma_\epsilon$ that is restricted to the $(u,v)$-phase plane
is a symmetry that only acts on the fibre:
\begin{equation}
\Gamma_\epsilon:(\tau,u(\tau),v(\tau))\mapsto(\tau,\hat{u}(u,v;\epsilon),\hat{v}(u,v;\epsilon)).
  \label{eq:symmetry_phase_plane}
\end{equation}
Moreover, its infinitesimal generator of the Lie group $X$ lacks an infinitesimal in the $\tau$-direction, i.e. $\xi\equiv 0$, and it is given by
\begin{equation}
X = \eta_1(u,v)\partial_u+\eta_2(u,v)\partial_v.
  \label{eq:generator_phase_plane}
\end{equation}
\label{def:symmetry_fibre_phase_plane}
\end{definition}
\hrule
$\;$\\
\noindent Next, we demonstrate that the calculations of these fibre preserving symmetries of time-invariant models restricted to the phase plane are straightforward. This is due to the fact that the two linearised symmetry conditions in Equation \eqref{eq:lin_sym_ori}, are condensed into a single solvable PDE (Thm \ref{thm:lin_sym_cond_fibre_phase_plane}) in this case. 
\hrule
\begin{theorem}[The linearised symmetry condition of fibre preserving symmetries of time-invariant models restricted to the phase plane.]
  Consider the time-invariant system of ODEs in Equation \eqref{eq:sys_auto}. Further, let $\Gamma_\epsilon$ be a fibre preserving symmetry of this model that is restricted to the $(u,v)$-phase plane according to Definition \ref{def:symmetry_fibre_phase_plane} and let the corresponding infinitesimal generator of the Lie group $X$ be given by Equation \eqref{eq:generator_phase_plane}. Then, the infinitesimals $\eta_1(u,v)$ and $\eta_2(u,v)$ defining these symmetries solve the single PDE given by
\begin{equation}
 \omega_1^2 \pdv{\eta_2}{u}+\omega_1\omega_2\left(\pdv{\eta_2}{v}-\pdv{\eta_1}{u}\right)-\omega_2^2\pdv{\eta_1}{v}=\left(\pdv{\omega_2}{u}\omega_1-\omega_2\pdv{\omega_1}{u}\right)\eta_1+\left(\pdv{\omega_2}{v}\omega_1-\omega_2\pdv{\omega_1}{v}\right)\eta_2.
\label{eq:lin_sym_fibre}
\end{equation}

\label{thm:lin_sym_cond_fibre_phase_plane}
\end{theorem}
  \dotfill
\begin{proof}
  The dynamics in the $(u,v)$-phase plane of Equation \eqref{eq:sys_auto} is described by a single first order ODE:
\begin{equation}
\dv{v}{u}=\Omega(u,v)=\dfrac{\omega_2(u,v)}{\omega_1(u,v)}.
  \label{eq:sys_phase}
\end{equation}
Moreover, the total derivative for the phase plane is given by $D_u=\partial_u+(\dd v/\dd u)\partial_v$, and the linearised symmetry condition according to Equation \eqref{eq:lin_sym} is given by
\begin{equation}
  D_u\eta_2-\Omega D_u\eta_1=\pdv{\Omega}{u}\eta_1+\pdv{\Omega}{v}\eta_2\quad\textrm{whenever}\quad\dv{v}{u}=\Omega(u,v).
  \label{eq:lin_sym_phase}
  \end{equation}
The left hand side of the above equation can be written as:
\begin{equation*}
  D_u\eta_2-\Omega D_u\eta_1=\pdv{\eta_2}{u}+\Omega\left(\pdv{\eta_2}{v}-\pdv{\eta_1}{u}\right)-\Omega^2\pdv{\eta_1}{v}\quad\textrm{whenever}\quad\dv{v}{u}=\Omega(u,v).
\end{equation*}
The partial derivatives in the right hand side of Equation \eqref{eq:lin_sym_phase} are given by
\begin{align*}
  \pdv{\Omega}{u}&=\dfrac{\pdv{\omega_2}{u}\omega_1-\omega_2\pdv{\omega_1}{u}}{\omega_1^{2}},\\
  \pdv{\Omega}{v}&=\dfrac{\pdv{\omega_2}{v}\omega_1-\omega_2\pdv{\omega_1}{v}}{\omega_1^{2}}.
\end{align*}
By plugging in these partial derivatives into the right hand side of Equation \eqref{eq:lin_sym_phase}, equating the resulting expression with the left hand side and lastly multiplying the resulting equation with the factor $\omega_1^{2}$ results in the PDE in Equation \eqref{eq:lin_sym_fibre}. 
\end{proof}
\dotfill\\
\hrule
$\;$\\Morover, a subgroup of time-invariant models as in Equation \eqref{eq:sys_auto} that are common in mathematical biology are definied by polynomial reaction terms $\omega_1$ and $\omega_2$ respectively. In this case the linearised symmetry condition in Equation \eqref{eq:lin_sym_fibre} decomposes into a system of coupled first order PDEs where the number of equations depends on the degree of the reaction terms.\\
\hrule
\begin{corollary}[An algorithm for solving the linearised symmetry condition in the case of polynomial reaction terms.]
  Consider the time-invariant system of first order ODEs in Equation \eqref{eq:sys_auto} where the reaction terms $\omega_1(u,v)$ and $\omega_2(u,v)$ are two polynomials of degree $d_1$ and $d_2$ respectively. Furthermore, assume that two polynomial ans\"atze are used for the infinitesimals $\eta_1(u,v)$ and $\eta_2(u,v)$ where the degree of these ans\"atze are $d_3$ and $d_4$ respectively. Then, the linearised symmetry condition in Equation \eqref{eq:lin_sym_fibre} decomposes into a system of $d\in\mathbb{N}_+$ non-linear algebraic equations for the unknown coefficients in the polynomial ans\"atze for the infinitesimals $\eta_1(u,v)$ and $\eta_2(u,v)$ where the number of equations are bounded by $0\leq d\leq D$ where the upper bound is given by
  \begin{equation}
    D={2+\delta\choose \delta}=\dfrac{(2+\delta)!}{2!\delta!},\quad\delta=\max\left(2\,\max\left(d_1,d_2\right),d_3,d_4\right).
    \label{eq:num_monomials}
    \end{equation}

\label{thm:lin_sym_cond_fibre_phase_plane_polynomial}
\end{corollary}
\dotfill
\begin{proof}
In the case when the reaction terms $\omega_1(u,v)$ and $\omega_2(u,v)$ are polynomials of degree $d_1$ and $d_2$ respectively and where two polynomial ans\"atze of degree $d_3$ and $d_4$ are used for the infinitesimals $\eta_1(u,v)$ and $\eta_2(u,v)$, the linearised symmetry condition in Equation \eqref{eq:lin_sym_fibre} corresponds to finding the roots of a multivariate polynomial in two varibles, namely $u$ and $v$. The degree of this polynomial will either be determined by the term $\omega_1^2$, the term $\omega_2^2$ or the degrees $d_3$ and $d_4$. Therefore, the number of monomials composed of $u$ and $v$ in this polynomial is given by Equation \eqref{eq:num_monomials}. Since these monomials are linearly independent it follows that all their coefficients in Equation \eqref{eq:lin_sym_fibre} equal zero. Lastly, all coefficients of the monomials composed of $u$ and $v$ are non-linear equations involving the unknown constants in the polynomial ans\"atze for $\eta_1(u,v)$ and $\eta_2(u,v)$, and hence the claim of the corollary follows.
\end{proof}
\dotfill\\
\hrule

\begin{remark}
  Note that the family of trivial generators are given by
  $$X_0=f(u,v)\left[\omega_1(u,v)\partial_u+\omega_2(u,v)\partial_v\right]$$
  where $f$ is an arbitrary function \cite{bluman1989symmetries}. Thus, if we choose the degrees $d_3$ and $d_4$ to be larger than $\max\left(d_1,d_2\right)$, we are guaranteed to find trivial generators.
\end{remark}
\hrule
\noindent $\;$\\Subsequently, we will try to solve this linearised symmetry condition for the three models that were presented previously starting with the LV-model. 
%--------------------------------------------------------------------------------------
%--------------------------------------------------------------------------------------
% Lotka-Volterra model
\section{The symmetries of the LV-model}
We will now consider the symmetries of the LV-model \cite{lotka1920undamped,lotka1925elements,volterra1926variations} given by
\begin{equation}
  \begin{split}
    \dv{u}{\tau}&=u(1-v),\\
    \dv{v}{\tau}&=\alpha v(u-1),\\    
    \end{split}
  \label{eq:LV}
\end{equation}
where the dynamics in the $(u,v)$-phase plane is determined by
\begin{equation}
\dv{v}{u}=\dfrac{\alpha v(u-1)}{u(1-v)}.
  \label{eq:LV_phase_plane}
\end{equation}
Since this model is time-invariant and separable its symmetries are given by (Thm \ref{thm:separable})
\begin{align}
  X_0&=\partial_\tau+u(1-v)\partial_u+\alpha v(u-1)\partial_v,\label{eq:LV_0}\\
  X_\tau&=\partial_\tau,\label{eq:LV_tau}\\
  X_u&=\dfrac{u}{u-1}\partial_u,\label{eq:LV_u}\\
  X_v&=\dfrac{\alpha v}{1-v}\partial_v.\label{eq:LV_v}
\end{align}
The symmetry of generated by $X_\tau$ in Equation \eqref{eq:LV_tau} corresponds to $\tau$-translations given by
\begin{equation}
\Gamma^{\mathrm{LV},\tau}_{\epsilon}:(\tau,u,v)\mapsto (\tau+\epsilon,u,v).
\end{equation}
For the other two non-trivial generators, i.e. $X_u$ in Equation \eqref{eq:LV_u} and $X_v$ in Equation \eqref{eq:LV_v}, we cannot write down an explicit equation for the symmetry. However, we can write down an implicit expression in terms of their canonical coordinates
\begin{align}
\Gamma^{\mathrm{LV},u}_{\epsilon}:&(s,r_1,r_2)\mapsto(s+\epsilon,r_1,r_2),\quad s=u-\ln(u),r_1=\tau,r_2=v,\label{eq:LV_u}\\
\Gamma^{\mathrm{LV},v}_{\epsilon}:&(s,r_1,r_2)\mapsto(s+\epsilon,r_1,r_2),\quad s=\ln\left(v^{1/\alpha}\right)-\dfrac{v}{\alpha},r_1=\tau,r_2=u.\label{eq:LV_v}\\
\end{align}
In practice, this means that we obtain the transformed coordinates $\hat{u}(\epsilon)$ of $\Gamma^{\mathrm{LV},u}_{\epsilon}$ in Equation \eqref{eq:LV_u} and $\hat{v}(\epsilon)$ of $\Gamma^{\mathrm{LV},v}_{\epsilon}$ in Equation \eqref{eq:LV_v} by solving the following two equations
\begin{align}
u-\ln(u)+\epsilon&=\hat{u}-\ln(\hat{u}),\label{eq:LV_u_implicit}\\
\ln\left(v^{1/\alpha}\right)-\dfrac{v}{\alpha}+\epsilon&=\ln\left(\hat{v}^{1/\alpha}\right)-\dfrac{\hat{v}}{\alpha},\label{eq:LV_v_implicit}
\end{align}
for $\hat{u}$ and $\hat{v}$ respectively. The action of these unidirectional symmetries is illustrated below (Fig \ref{fig:LV_symmetries}). To interpret the symmetries of the LV model it is of interest to analyse their respective invariants.


\begin{figure}[htbp!]
  \begin{center}
\includegraphics[width=\textwidth]{LV_symmetries}
\caption{\textit{Unidirectional symmetries of the LV-model}. The original solution is defined by $\alpha=1$, the initial conditions $(u_0,v_0)=(1.00,0.10)$ and the solutions are then transformed with a transformation parameter of $\epsilon=0.5$. (\textbf{A}) The action of the time translation symmetry $\Gamma^{\mathrm{LV},\tau}_{\epsilon}$. (\textbf{B}) The action of the $u$-directional symmetry $\Gamma^{\mathrm{LV},u}_{\epsilon}$ which gives rise to a solution curve with an internal energy of $H+\alpha\epsilon$. (\textbf{C}) The action of the $v$-directional symmetry $\Gamma^{\mathrm{LV},v}_{\epsilon}$ which gives rise to a solution curve with an internal energy of $H-\alpha\epsilon$.}
\label{fig:LV_symmetries}
\end{center}
\end{figure}

The invariants of the trivial symmetry corresponds to solving the ODE in the $(u,v)$-phase plane in Equation \eqref{eq:LV_phase_plane}. Since this ODE is separable, it is straightforward to solve and its phase trajectories are given by \cite{murray2002}
\begin{equation}
  H=\alpha u+v-\ln\left(u^\alpha v\right),\quad H=u_0+v_0-\ln\left(u_0^\alpha v_0\right)
  \label{eq:phase_trajectory_LV}
\end{equation}
where $u_0$ and $v_0$ correspond to the initial conditions defining the particular solution curve. Here, $H$ is an invariant of the trivial infinitesimal generator $X_0$ in Equation \eqref{eq:LV_0}, and it should be interpreted as the energy of a solution trajectory which in fact corresponds to a conservation law of the LV-model \cite{murray2002}. Moreover, given Equation \eqref{eq:phase_trajectory_LV} for a closed solution trajectory in the $(u,v)$-phase plane, it is possible to formulate the action of the symmetries $\Gamma^{\mathrm{LV},u}_\epsilon$ in Equation \eqref{eq:LV_u} and $\Gamma^{\mathrm{LV},v}_\epsilon$ in Equation \eqref{eq:LV_v} in terms of the transformed solution curve. In both cases, the solution curves that are obtained after transforming the original solution curve in Equation \eqref{eq:phase_trajectory_LV} are given by the following equations
\begin{align}
  \Gamma_{\epsilon}^{\mathrm{LV},u}:H+\alpha\epsilon=\alpha u+v-\ln\left(u^\alpha v\right),\quad H=u_0+v_0-\ln\left(u_0^\alpha v_0\right),\label{eq:trans_LV_u}\\
  \Gamma_{\epsilon}^{\mathrm{LV},v}:H-\alpha\epsilon=\alpha u+v-\ln\left(u^\alpha v\right),\quad H=u_0+v_0-\ln\left(u_0^\alpha v_0\right).\label{eq:trans_LV_v}
\end{align}
From this equation, we can conclude that transformation by both these symmetries of the LV-model correspond to changing the initial conditions $u_0$ and $v_0$ defining a particular solution curve and the parameter $\alpha$ is preserved under the action of these symmetries. Next, we wish to calculate the invariants of the non-trivial symmetries generated by $X_u$ in Equation \eqref{eq:LV_u} and $X_v$ in Equation \eqref{eq:LV_v} in order to interpret their biological meaning.


% --------------------------------------------------------------------------------------
%--------------------------------------------------------------------------------------
% The SIR model
\section{The symmetries of the SIR-model}
We remind ourselves that we want to study the SIR model
\begin{equation*}
  \begin{split}
    \dv{S}{t}&=-rSI,\\
    \dv{I}{t}&=rSI-aI.\\
    \dv{R}{t}&=aI.\\    
    \end{split}
  \end{equation*}
  and we are looking for a generator of the following kind:
\begin{equation}
X=\xi(t,S,I,R)\partial_t+\eta_1(t,S,I,R)\partial_S+\eta_2(t,S,I,R)\partial_I+\eta_3(t,S,I,R)\partial_R.
\end{equation}
The first linearised symmetry condition is given by the following equation:
\begin{align}
  - a r I^{2} S \pdv{\xi}{I} + a r I^{2} S \pdv{\xi}{R} - a I \pdv{\eta_1}{I} + a I \pdv{\eta_1}{R}&\nonumber\\
  + r^{2} I^{2} S^{2} \pdv{\xi}{I} - r^{2} I^{2} S^{2} \pdv{\xi}{S} + r I S \pdv{\eta_1}{I} - r I S \pdv{\eta_1}{S}&\nonumber\\
  + r I S \pdv{\xi}{t} + r I\eta_1 + r S\eta_2 + \pdv{\eta_1}{t}&=0.\label{eq:SIR_lin_sym_1}
\end{align}
The second linearised symmetry condition is given by the following equation:
\begin{align}
  - a^{2} I^{2} \pdv{\xi}{I} + a^{2} I^{2} \pdv{\xi}{R} + 2 a r I^{2} S \pdv{\xi}{I} - a r I^{2} S \pdv{\xi}{R} - a r I^{2} S \pdv{\xi}{S}&\nonumber\\
  - a I \pdv{\eta_2}{I} + a I \pdv{\eta_2}{R} + a I \pdv{\xi}{t} + a\eta_2&\nonumber\\
  - r^{2} I^{2} S^{2} \pdv{\xi}{I} + r^{2} I^{2} S^{2} \pdv{\xi}{S} + r I S \pdv{\eta_2}{I} - r I S \pdv{\eta_2}{S}&\nonumber\\
  - r I S \pdv{\xi}{t} - r I\eta_1 - r S\eta_2 + \pdv{\eta_2}{t}&=0.\label{eq:SIR_lin_sym_2}
\end{align}
The third linearised symmetry condition is given by the following equation:
\begin{align}
  a^{2} I^{2} \pdv{\xi}{I} - a^{2} I^{2} \pdv{\xi}{R} - a r I^{2} S \pdv{\xi}{I} + a r I^{2} S \pdv{\xi}{S}&\nonumber\\
  - a I \pdv{\eta_3}{I} + a I \pdv{\eta_3}{R} - a I \pdv{\xi}{t} - a\eta_2&\nonumber\\
  + r I S \pdv{\eta_3}{I} - r I S \pdv{\eta_3}{S} + \pdv{\eta_3}{t}&=0.\label{eq:SIR_lin_sym_3}
\end{align}
\subsection{Ans\"atze assuming that the equations for the tangents are independent of the states}
The assumption that the derivatives of the tangents are at most functions of the time $t$ amounts to finding the roots of a polynomial of the states $u$ and $v$. Since these monomials are linearly independent we obtain the following equations:
\begin{align}
1:&\pdv{\eta_1}{t}=0,\\
I:&- a \pdv{\eta_1}{I} + a \pdv{\eta_1}{R} + r\eta_1=0,\\
S:&r\eta_2=0,\\
S I:&r \pdv{\eta_1}{I} - r \pdv{\eta_1}{S} + r \pdv{\xi}{t}=0,\\
S I^{2}:&- a r \pdv{\xi}{I} + a r \pdv{\xi}{R}=0,\\
S^{2} I^{2}:&r^{2} \pdv{\xi}{I} - r^{2} \pdv{\xi}{S}=0,
\end{align}
for the first linearised symmetry condition, the following equations
\begin{align}
1:&a\eta_2 + \pdv{\eta_2}{t}=0,\\
I:&- a \pdv{\eta_2}{I} + a \pdv{\eta_2}{R} + a \pdv{\xi}{t} - r\eta_1=0,\\
I^{2}:&- a^{2} \pdv{\xi}{I} + a^{2} \pdv{\xi}{R}=0,\\
S:&- r\eta_2=0,\\
S I:&r \pdv{\eta_2}{I} - r \pdv{\eta_2}{S} - r \pdv{\xi}{t}=0,\\
S I^{2}:&2 a r \pdv{\xi}{I} - a r \pdv{\xi}{R} - a r \pdv{\xi}{S}=0,\\
S^{2} I^{2}:&- r^{2} \pdv{\xi}{I} + r^{2} \pdv{\xi}{S}=0,
\end{align}
for the second linearised symmetry condition and the following equations
\begin{align}
1:&- a\eta_2 + \pdv{\eta_3}{t}=0,\\
I:&- a \pdv{\eta_3}{I} + a \pdv{\eta_3}{R} - a \pdv{\xi}{t}=0,\\
I^{2}:&a^{2} \pdv{\xi}{I} - a^{2} \pdv{\xi}{R}=0,\\
S I:&r \pdv{\eta_3}{I} - r \pdv{\eta_3}{S}=0,\\
S I^{2}:&- a r \pdv{\xi}{I} + a r \pdv{\xi}{S}=0,
\end{align}
for the third linearised symmetry condition.
%--------------------------------------------------------------------------------------
%--------------------------------------------------------------------------------------
% The BZ-model
\section{The symmetries of the BZ-model}
We remind ourselves that we want to study the BZ model

\begin{equation}
  \begin{split}
    \dv{u}{t}&=\dfrac{1}{\varepsilon}v-\dfrac{1}{\varepsilon}\left(\dfrac{1}{3}u^3-u\right),\\
    \dv{v}{t}&=-u.\\    
    \end{split}
  \label{eq:BZ}
\end{equation}
and we are looking for a generator of the following kind:
\begin{equation}
X=\xi(t,u,v)\partial_t+\eta_1(t,u,v)\partial_u+\eta_2(t,u,v)\partial_v.
\end{equation}
The linearised symmetry condition 1 yields the following equations:


\begin{align}
  - u \frac{d}{d v} \eta_1 + \pdv{\eta_1}{t} + \frac{\eta_1 u^{2}}{\varepsilon} - \frac{\eta_1}{\varepsilon} - \frac{\eta_2}{\varepsilon} - \frac{u^{4} \frac{d}{d v} \xi}{3 \varepsilon}&\nonumber\\
  - \frac{u^{3} \frac{d}{d u} \eta_1}{3 \varepsilon} + \frac{u^{3} \pdv{\xi}{t}}{3 \varepsilon} + \frac{u^{2} \frac{d}{d v} \xi}{\varepsilon} + \frac{u v \frac{d}{d v} \xi}{\varepsilon}&\nonumber\\
  + \frac{u \frac{d}{d u} \eta_1}{\varepsilon} - \frac{u \pdv{\xi}{t}}{\varepsilon} + \frac{v \frac{d}{d u} \eta_1}{\varepsilon} - \frac{v\pdv{\xi}{t}}{\varepsilon}&\nonumber\\
  - \frac{u^{6} \frac{d}{d u} \xi}{9 \varepsilon^{2}} + \frac{2 u^{4} \frac{d}{d u} \xi}{3 \varepsilon^{2}} + \frac{2 u^{3} v \frac{d}{d u} \xi}{3 \varepsilon^{2}} - \frac{u^{2} \frac{d}{d u} \xi}{\varepsilon^{2}}&\nonumber\\
  - \frac{2 u v \frac{d}{d u} \xi}{\varepsilon^{2}} - \frac{v^{2} \frac{d}{d u} \xi}{\varepsilon^{2}}&=0\label{eq:BZ_lin_sym_1}
\end{align}
and we see that this equation amounts to finding the roots of a polynomial of the states $u$ and $v$. Since these monomials are linearly independent we obtain the following determining equations:
\begin{align}
  1:&\pdv{\eta_1}{t} - \frac{1}{\varepsilon}\left(\eta_1 - \eta_2\right)=0,\\
  v:&\pdv{\eta_1}{u} - \pdv{\xi}{t}=0,\\
  v^{2}:&- \pdv{\xi}{u}=0,\\
  u:&-\pdv{\eta_1}{v} + \frac{1}{\varepsilon}\left(\pdv{\eta_1}{u}  - \pdv{\xi}{t}\right)=0,\\  
  u v:&\varepsilon\pdv{\xi}{v} - 2 \pdv{\xi}{u}=0,\\
  u^{2}:&\eta_1 + \pdv{\xi}{v} - \frac{1}{\varepsilon} \pdv{\xi}{u}=0,\\
  u^{3}:&- \pdv{\eta_1}{u} + \pdv{\xi}{t}=0,\\
u^{3} v:&\pdv{\xi}{u}=0,\\
 u^{4}:&- \frac{1}{3}\pdv{\xi}{v} + \frac{2}{3 \varepsilon}\pdv{\xi}{u}=0,\\
 u^{6}:&- \pdv{\xi}{u}=0.\\
\end{align}

Similarly, the second linearised symmetry condition is given by
\begin{align}
\eta_1 - u^{2} \pdv{}{v} \xi - u \pdv{}{v} \eta_2 + u \pdv{\xi}{t} + \pdv{\eta_2}{t} - \frac{u^{4} \pdv{}{u} \xi}{3 \varepsilon} - \frac{u^{3} \pdv{}{u} \eta_2}{3 \varepsilon} + \frac{u^{2} \pdv{}{u} \xi}{\varepsilon} + \frac{u v \pdv{}{u} \xi}{\varepsilon} + \frac{u \pdv{}{u} \eta_2}{\varepsilon} + \frac{v \pdv{}{u} \eta_2}{\varepsilon}&=0\label{eq:BZ_lin_sym_1}
\end{align}
and the corresponding determining equations

\begin{align*}
1:&\eta_1 + \pdv{\eta_2}{t}=0\\
v:&\pdv{\eta_2}{u}=0\\
u:&- \pdv{\eta_2}{v}  + \pdv{\xi}{t} + \frac{1}{\varepsilon}\pdv{\eta_2}{u} =0\\
u v:&\pdv{\xi}{u}=0\\
u^{2}:&- \pdv{\xi}{v} + \frac{1}{\varepsilon}\pdv{\xi}{u} =0\\
u^{3}:&- \pdv{\eta_2}{u}=0\\
u^{4}:&- \pdv{\xi}{u}=0\\
\end{align*}
% --------------------------------------------------------------------------------------
%--------------------------------------------------------------------------------------
% 
\section{The symmetries of the Brusselator}
The Brusselator \cite{prigogine1968symmetry}...

%--------------------------------------------------------------------------------------
% THE BIBLIOGRAPHY
%--------------------------------------------------------------------------------------
\clearpage
\bibliographystyle{unsrt}
\bibliography{symmetry_bib}
%--------------------------------------------------------------------------------------
% THE DOCUMENT ENDS
%--------------------------------------------------------------------------------------
\end{document}