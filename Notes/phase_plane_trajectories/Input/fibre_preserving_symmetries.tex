A common class of models in Mathematical Biology is that of two state time-invariant models. For example, these models typically describe the evolution of two competing populations or the evolution of two reacting proteins. To this end, we will now consider the following autonomous (specifically time-invariant) two state system of first order ODEs
\begin{equation}
  \begin{split}
    \dv{u}{\tau}&=\omega_1(u,v),\\
    \dv{v}{\tau}&=\omega_2(u,v).
  \end{split}
  \label{eq:sys_auto}
\end{equation}
where the right hand sides $\omega_1(u,v)$ and $\omega_2(u,v)$ referred to as the \textit{reaction terms} are independent of the time $\tau$. For these types of systems, there are two well-known symmetries. The first one is the time translation symmetry 

\begin{equation}
  \begin{split}
    \Gamma_{\epsilon}&:(\tau,u,v)\mapsto(\tau+\epsilon,u,v)\\
   \textrm{generated by}&\\
    X&=\partial_\tau
  \end{split}
  \label{eq:time_trans}
\end{equation}
and this symmetry is common to all autonomous models. The second symmetry is the \textit{trivial symmetry} generated by the infinitesimal generator of the Lie group given by
\begin{equation}
X=\partial_\tau+\omega_1(u,v)\partial_u+\omega_2(u,v)\partial_v
  \label{eq:trivial}
\end{equation}
and this vector field is parallel to the vector field of the original system of ODEs in Equation \eqref{eq:sys_auto}. Consequently, this symmetry maps points on a solution curve onto other points on the \textit{same solution curve}. Note that the infinitesimals in the time direction $\xi$ of these two generators are \textit{independent of the states} $u$ and $v$ meaning that these generators do not mix the time and state dependence. Such symmetries are called \textit{fibre preserving}, and next we are interested in such symmetries that are restricted to the $(u,v)$-phase plane.\\

\hrule
\begin{definition}[Fibre preserving symmetries of time-invariant models restricted to the phase plane]
Consider the autonomous system of ODEs in Equation \eqref{eq:sys_auto}. A \textit{fibre-preserving symmetry} $\Gamma_\epsilon$ that is restricted to the $(u,v)$-phase plane
is a symmetry that only acts on the fibre:
\begin{equation}
\Gamma_\epsilon:(\tau,u(\tau),v(\tau))\mapsto(\tau,\hat{u}(u,v;\epsilon),\hat{v}(u,v;\epsilon)).
  \label{eq:symmetry_phase_plane}
\end{equation}
Moreover, its infinitesimal generator of the Lie group $X$ lacks an infinitesimal in the $\tau$-direction, i.e. $\xi\equiv 0$, and it is given by
\begin{equation}
X = \eta_1(u,v)\partial_u+\eta_2(u,v)\partial_v.
  \label{eq:generator_phase_plane}
\end{equation}
\label{def:symmetry_fibre_phase_plane}
\end{definition}
\hrule
$\;$\\
\noindent Next, we demonstrate that the calculations of these fibre preserving symmetries of time-invariant models restricted to the phase plane are straightforward. This is due to the fact that the two linearised symmetry conditions in Equation \eqref{eq:lin_sym_ori}, are condensed into a single solvable PDE (Thm \ref{thm:lin_sym_cond_fibre_phase_plane}) in this case. 
\hrule
\begin{theorem}[The linearised symmetry condition of fibre preserving symmetries of time-invariant models restricted to the phase plane.]
  Consider the time-invariant system of ODEs in Equation \eqref{eq:sys_auto}. Further, let $\Gamma_\epsilon$ be a fibre preserving symmetry of this model that is restricted to the $(u,v)$-phase plane according to Definition \ref{def:symmetry_fibre_phase_plane} and let the corresponding infinitesimal generator of the Lie group $X$ be given by Equation \eqref{eq:generator_phase_plane}. Then, the infinitesimals $\eta_1(u,v)$ and $\eta_2(u,v)$ defining these symmetries solve the single PDE given by
\begin{equation}
 \omega_1^2 \pdv{\eta_2}{u}+\omega_1\omega_2\left(\pdv{\eta_2}{v}-\pdv{\eta_1}{u}\right)-\omega_2^2\pdv{\eta_1}{v}=\left(\pdv{\omega_2}{u}\omega_1-\omega_2\pdv{\omega_1}{u}\right)\eta_1+\left(\pdv{\omega_2}{v}\omega_1-\omega_2\pdv{\omega_1}{v}\right)\eta_2.
\label{eq:lin_sym_fibre}
\end{equation}

\label{thm:lin_sym_cond_fibre_phase_plane}
\end{theorem}
  \dotfill
\begin{proof}
  The dynamics in the $(u,v)$-phase plane of Equation \eqref{eq:sys_auto} is described by a single first order ODE:
\begin{equation}
\dv{v}{u}=\Omega(u,v)=\dfrac{\omega_2(u,v)}{\omega_1(u,v)}.
  \label{eq:sys_phase}
\end{equation}
Moreover, the total derivative for the phase plane is given by $D_u=\partial_u+(\dd v/\dd u)\partial_v$, and the linearised symmetry condition according to Equation \eqref{eq:lin_sym} is given by
\begin{equation}
  D_u\eta_2-\Omega D_u\eta_1=\pdv{\Omega}{u}\eta_1+\pdv{\Omega}{v}\eta_2\quad\textrm{whenever}\quad\dv{v}{u}=\Omega(u,v).
  \label{eq:lin_sym_phase}
  \end{equation}
The left hand side of the above equation can be written as:
\begin{equation*}
  D_u\eta_2-\Omega D_u\eta_1=\pdv{\eta_2}{u}+\Omega\left(\pdv{\eta_2}{v}-\pdv{\eta_1}{u}\right)-\Omega^2\pdv{\eta_1}{v}\quad\textrm{whenever}\quad\dv{v}{u}=\Omega(u,v).
\end{equation*}
The partial derivatives in the right hand side of Equation \eqref{eq:lin_sym_phase} are given by
\begin{align*}
  \pdv{\Omega}{u}&=\dfrac{\pdv{\omega_2}{u}\omega_1-\omega_2\pdv{\omega_1}{u}}{\omega_1^{2}},\\
  \pdv{\Omega}{v}&=\dfrac{\pdv{\omega_2}{v}\omega_1-\omega_2\pdv{\omega_1}{v}}{\omega_1^{2}}.
\end{align*}
By plugging in these partial derivatives into the right hand side of Equation \eqref{eq:lin_sym_phase}, equating the resulting expression with the left hand side and lastly multiplying the resulting equation with the factor $\omega_1^{2}$ results in the PDE in Equation \eqref{eq:lin_sym_fibre}. 
\end{proof}
\dotfill\\
\hrule
$\;$\\Morover, a sub group of time-invariant models as in Equation \eqref{eq:sys_auto} that are common in mathematical biology are definied by polynomial reaction terms $\omega_1$ and $\omega_2$ respectively. In this case the linearised symmetry condition in Equation \eqref{eq:lin_sym_fibre} decomposes into a system of coupled first order PDEs where the number of equations depends on the degree of the reaction terms.

In fact, we can characterise all symmetries as either trivial or non-trivial. The trivial symmetries map points on a solution curve to \textit{the same solution curve}, and infinitesimally this corresponds to the infinitesimals, i.e. the local action of the symmetry, being parallel to the vector field of the ODE itself. If the infinitesimals are not parallel to the vector field of the ODE itself, the symmetries map a solution curve to another distinct solution curve and in this case the symmetries are referred to as non-trivial. Next, we will prove the condition defining whether a fibre-preserving symmetry restricted to the phase plane is trivial or not. 

\hrule
\begin{theorem}[Trivial and non-trivial fibre-preserving symmetries of time invariant models restricted to the phase plane.]
  Consider the time-invariant system of ODEs in Equation \eqref{eq:sys_auto}. Further, let $\Gamma_\epsilon$ be a fibre preserving symmetry of this model that is restricted to the $(u,v)$-phase plane according to Definition \ref{def:symmetry_fibre_phase_plane} and let the corresponding infinitesimal generator of the Lie group $X$ be given by Equation \eqref{eq:generator_phase_plane}. Then, the symmetry is \textit{trivial} if the infinitesimals $\eta_1(u,v)$ and $\eta_2(u,v)$ satisfy
  \begin{equation}
    \omega_1(u,v)\eta_2(u,v)\equiv\omega_2(u,v)\eta_1(u,v)
    \label{eq:trivial_symmetry}
    \end{equation}
    in addition to solving Equation \eqref{eq:lin_sym_fibre}. Conversely, the symmetry is non-trivial if it satisfies
  \begin{equation}
    \omega_1(u,v)\eta_2(u,v)\neq\omega_2(u,v)\eta_1(u,v)
    \label{eq:nontrivial_symmetry}
    \end{equation}    
in addition to Equation \eqref{eq:lin_sym_fibre}.
\label{thm:trivial_and_nontrivial}
\end{theorem}
  \dotfill
\begin{proof}
  The dynamics of the $(u,v)$-phase plane is governed by Equation \eqref{eq:sys_phase}. Locally, any solution to this equation at a point $(u,v)$ in the phase plane travels with the vector
  $$\vec{v}_1=\begin{pmatrix}1 \\ \dd v/\dd u\end{pmatrix}=\begin{pmatrix}1 \\ \Omega(u,v)\end{pmatrix}$$
  where $\Omega(u,v)=\omega_2(u,v)/\omega_1(u,v)$. The local action of the symmetry at the same point $(u,v)$ in the phase plane is determined by the vector
  $$\vec{v}_2=\begin{pmatrix}\eta_1(u,v) \\ \eta_2(u,v)\end{pmatrix}.$$
  Now, consider the matrix M
  $$M=\begin{pmatrix}\vec{v}_1&\vec{v}_2\end{pmatrix}=\begin{pmatrix}1&\eta_1(u,v)\\\Omega(u,v)&\eta_2(u,v)\end{pmatrix}$$
and its determinant
  \begin{equation}
    \overline{Q}=\det\left(M\right)=\eta_2(u,v)-\Omega(u,v)\eta_1(u,v).
    \label{eq:characteristic}
    \end{equation}
  Then, the two vectors are parallel if $\overline{Q}=0$ and non-parallel if $\overline{Q}\neq 0$.
  
\end{proof}
\dotfill\\
\hrule
\begin{remark}
The inner product $\overline{Q}$ in Equation \eqref{eq:characteristic} is referred to as the \textit{reduced characteristic} \cite{hydon2000symmetry}.
\end{remark}
\begin{remark}
From Equation \eqref{eq:nontrivial_symmetry}, it is clear that the choice $\eta_1(u,v)=\omega_1(u,v)$ and $\eta_2(u,v)=\omega_2(u,v)$ results in a trivial symmetry. 
\end{remark}
\begin{remark}
From Equation \eqref{eq:nontrivial_symmetry}, it is clear that uni-directional symmetries, i.e. $\eta_1\neq 0$ and $\eta_2=0$ or $\eta_1=0$ and $\eta_2\neq 0$, are always non-trivial. 
\end{remark}
\hrule
  $\;$\\\noindent
  In the case of \textit{separable models}, two non-trivial symmetries in addition to the time translation symmetry are known. In fact, these non-trivial symmetries are uni-directional in the $u$ and $v$ directions respectively (Thm \ref{thm:separable}).

\hrule
\begin{theorem}[Non-trivial symmetries of separable ODEs]
  Consider the time-invariant system of ODEs in Equation \eqref{eq:sys_auto} where the reaction terms are separable:
  \begin{equation}
    \begin{split}
      \omega_1(u,v)&=g_1(u)f_1(v),\\
      \omega_2(u,v)&=g_2(u)f_2(v),\\      
    \end{split}
    \label{eq:sep_reac}
    \end{equation}
    where $g_1,g_2,f_1,f_2$ are continuous and differentiable functions. Further, let $\Gamma_\epsilon$ be a fibre preserving symmetry of this model that is restricted to the $(u,v)$-phase plane according to Definition \ref{def:symmetry_fibre_phase_plane} and let the corresponding infinitesimal generator of the Lie group $X$ be given by Equation \eqref{eq:generator_phase_plane}. Then, two infinitesimal generators of the Lie group corresponding to non-trivial symmetries are given by
    \begin{align}
      X_u&=\dfrac{g_1(u)}{g_2(u)}\partial_u,\label{eq:X_u}\\
      X_v&=\dfrac{f_2(v)}{f_1(v)}\partial_v.\label{eq:X_v}
     \end{align}

\label{thm:separable}
\end{theorem}
  \dotfill
\begin{proof}
  Starting with the generator $X_u$ in Equation \eqref{eq:X_u}, it holds that $\eta_1(u,v)=g_1(u)/g_2(u)$ and $\eta_2(u,v)=0$. Thus, in this case the linearised symmetry condition in Equation \eqref{eq:lin_sym_fibre} is given by
  \begin{equation}
    -\omega_1\omega_2\pdv{\eta_1}{u}=\left(\pdv{\omega_2}{u}\omega_1-\omega_2\pdv{\omega_1}{u}\right)\eta_1.
    \label{eq:lin_sym_u}
    \end{equation}
    The left hand side of Equation \eqref{eq:lin_sym_u} can be written as follows
    $$-\omega_1\omega_2\pdv{\eta_1}{u}=-g_1(u)f_1(v)g_2(u)f_2(v)\left(\dfrac{g_1'(u)g_2(u)-g_1(u)g_2'(u)}{g_2(u)^2}\right).$$
    The right hand side of Equation \eqref{eq:lin_sym_u} can be written as follows
    \begin{align*}
      \left(\pdv{\omega_2}{u}\omega_1-\omega_2\pdv{\omega_1}{u}\right)\eta_1&=\left(g_2'(u)f_2(v)g_1(u)f_1(v)-g_2(u)f_2(v)g_1'(u)f_1(v)\right)\dfrac{g_1(u)}{g_2(u)}\\
      &=g_1(u)f_1(v)g_2(u)f_2(v)\left(\dfrac{g_1(u)g_2'(u)-g_1'(u)g_2(u)}{g_2(u)^2}\right)
      \end{align*}
      and thus the left hand side equals the right hand side. For the generator $X_v$ in Equation \eqref{eq:X_v}, it holds that $\eta_1(u,v)=0$ and $\eta_2(u,v)=f_2(v)/f_1(v)$. In this case, the linearised symmetry condition in Equation \eqref{eq:lin_sym_fibre} is given by
      \begin{equation}
        \omega_1\omega_2\pdv{\eta_2}{v}=\left(\pdv{\omega_2}{v}\omega_1-\omega_2\pdv{\omega_1}{v}\right)\eta_2
        \label{eq:lin_sym_v}
      \end{equation}
    The left hand side of Equation \eqref{eq:lin_sym_v} can be written as follows
    $$\omega_1\omega_2\pdv{\eta_2}{v}=g_1(u)f_1(v)g_2(u)f_2(v)\left(\dfrac{f_2'(v)f_1(v)-f_2(v)f_1'(v)}{f_1(u)^2}\right).$$
    The right hand side of Equation \eqref{eq:lin_sym_u} can be written as follows
    \begin{align*}
      \left(\pdv{\omega_2}{v}\omega_1-\omega_2\pdv{\omega_1}{v}\right)\eta_2&=\left(g_2(u)f_2'(v)g_1(u)f_1(v)-g_2(u)f_2(v)g_1(u)f_1'(v)\right)\dfrac{f_2(v)}{f_1(v)}\\
      &=g_1(u)f_1(v)g_2(u)f_2(v)\left(\dfrac{f_2'(v)f_1(v)-f_2(v)f_1'(v)}{f_1(u)^2}\right)
    \end{align*}
    and thus the left hand side equals the right hand side.
\end{proof}
\dotfill\\
\hrule
\begin{remark}
  This result follows directly from the properties of separable ODEs. It is well-known that the separable ODE
  $$\dv{y}{x}=f(x)g(x)$$
  has a symmetry generated by $X=1/f(x)\partial x$ \cite{stephani1989differential}.
  \end{remark}
\hrule
$\:$\\[0.1cm]\\
\noindent In summary, the separable time-invariant system of ODEs given by
\begin{equation}
  \begin{split}
    \dv{u}{\tau}&=g_1(u)f_1(v),\\
    \dv{v}{\tau}&=g_2(u)f_2(v),\\
  \end{split}
  \label{eq:sys_separable}
  \end{equation}
  has four symmetries generated by
    \begin{align}
      X_0&=\partial_\tau+g_1(u)f_1(v)\partial_u+g_2(u)f_2(v)\partial_v,\label{eq:gen_0}\\
      X_\tau&=\partial_\tau,\label{eq:gen_tau}\\
      X_u&=\dfrac{g_1(u)}{g_2(u)}\partial_u,\label{eq:gen_u}\\
      X_v&=\dfrac{f_2(v)}{f_1(v)}\partial_v.\label{eq:gen_v}
     \end{align}
     Here, the trivial symmetry is given by $X_0$ and all the non-trival symmetries $X_\tau$, $X_u$ and $X_v$ are unidirectional as they act solely in the $\tau$-, $u$- and $v$-direction respectively. Moreover, the non-trivial symmetries $X_u$ and $X_v$ are composed of rational functions of the states, and therefore it is reasonable to search for generators where the inifinitesimals are rational functions of the states in the case where we have non-separable but polynomial reaction terms. Next, we will present an algorithm based on this for finding the non-trivial symmetries of models with polynomial reaction terms.

\hrule
\begin{theorem}[An algorithm for solving the linearised symmetry condition in the case of polynomial reaction terms.]
  Consider the time-invariant system of first order ODEs in Equation \eqref{eq:sys_auto} where the reaction terms $\omega_1(u,v)$ and $\omega_2(u,v)$ are two polynomials of degree $d_1$ and $d_2$ respectively. Furthermore, assume that rational ans\"atze are used for the infinitesimals $\eta_1(u,v)$ and $\eta_2(u,v)$:
  \begin{align}
    \eta_1(u,v)&=\dfrac{P_{1,n}(u,v)}{P_{1,d}(u,v)},\label{eq:ansatz_1}\\
    \eta_2(u,v)&=\dfrac{P_{2,n}(u,v)}{P_{2,d}(u,v)}\label{eq:ansatz_2}    
   \end{align}
where the degrees of the polynomials in the numerators $P_{1,n}$ and $P_{2,n}$ are $d_{1,n}$ and $d_{2,n}$ respectively and the degrees of the polynomials in the denominators $P_{1,d}$ and $P_{2,d}$ are $d_{1,d}$ and $d_{2,d}$ respectively. Then, the linearised symmetry condition in Equation \eqref{eq:lin_sym_fibre} decomposes into a system of $d\in\mathbb{N}_+$ non-linear algebraic equations for the unknown coefficients in the polynomial ans\"atze for the infinitesimals $\eta_1(u,v)$ and $\eta_2(u,v)$ where the number of equations are bounded by $0\leq d\leq D$ where the upper bound is given by
\small
\begin{equation}
  \begin{split}
    D&={2+\delta-1\choose \delta}=\delta+1,\\
    \delta&=\max\left((2d_1+d_{2,n}+d_{2,d}),(2d_2+d_{1,n}+d_{1,d}),(d_1+d_2+d_{1,n}+d_{1,d}+2d_{2,d}),(d_1+d_2+2d_{1,d}+d_{2,d}+d_{2,n})\right).
    \end{split}
    \label{eq:num_monomials}
    \end{equation}

\label{thm:lin_sym_cond_fibre_phase_plane_polynomial}
\end{theorem}
\dotfill
\begin{proof}
 After plugging in the ans\"atze in Equation \eqref{eq:ansatz_1} and Equation \eqref{eq:ansatz_2}, the linearised symmetry condition becomes
 \begin{equation}
   \begin{split}
     &\omega_1^2 \pdv{}{u}\left(\frac{P_{2,n}}{P_{2,d}}\right)+\omega_1\omega_2\left(\pdv{}{v}\left(\frac{P_{2,n}}{P_{2,d}}\right)-\pdv{}{u}\left(\frac{P_{1,n}}{P_{1,d}}\right)\right)-\omega_2^2\pdv{}{v}\left(\frac{P_{1,n}}{P_{1,d}}\right)\\
     &=\left(\pdv{\omega_2}{u}\omega_1-\omega_2\pdv{\omega_1}{u}\right)\frac{P_{1,n}}{P_{1,d}}+\left(\pdv{\omega_2}{v}\omega_1-\omega_2\pdv{\omega_1}{v}\right)\frac{P_{2,n}}{P_{2,d}}.
     \end{split}
\label{eq:lin_sym_ansatz}
\end{equation}
and all the partial derivatives of the tangential ans\"atze will result in the terms $P_{1,d}^2$ and $P_{2,d}^2$ in the denominators on the left hand side. By multiplying Equation \eqref{eq:lin_sym_ansatz} by the factor $P_{1,d}^2 P_{2,d}^2$, the resulting equation is
 \begin{equation}
   \begin{split}
     &P_{1,d}^2 P_{2,d}^2\omega_1^2 \pdv{}{u}\left(\frac{P_{2,n}}{P_{2,d}}\right)+P_{1,d}^2 P_{2,d}^2\omega_1\omega_2\left(\pdv{}{v}\left(\frac{P_{2,n}}{P_{2,d}}\right)-\pdv{}{u}\left(\frac{P_{1,n}}{P_{1,d}}\right)\right)-P_{1,d}^2 P_{2,d}^2\omega_2^2\pdv{}{v}\left(\frac{P_{1,n}}{P_{1,d}}\right)\\
     &=\left(\pdv{\omega_2}{u}\omega_1-\omega_2\pdv{\omega_1}{u}\right)P_{1,n}P_{1,d} P_{2,d}^2+\left(\pdv{\omega_2}{v}\omega_1-\omega_2\pdv{\omega_1}{v}\right)P_{1,d}^2 P_{2,d}P_{2,n}.
     \end{split}
\label{eq:lin_sym_ansatz_modified}
\end{equation}
\noindent The first term on the left hand side is a polynomial of order $2d_1+d_{2,n}+d_{2,d}$ and the third term on the left hand side is a polynomial of order $2d_2+d_{1,n}+d_{1,d}$. The first term on the right hand side is a polynomial of order $d_1+d_2+d_{1,n}+d_{1,d}+2d_{2,d}$ and the second term is a polynomial or order $d_1+d_2+2d_{1,d}+d_{2,d}+d_{2,n}$. Therefore, Equation \eqref{eq:lin_sym_ansatz_modified} corresponds to finding the roots of a polynomial in $u$ and $v$ where the number of monomials is given by Equation \eqref{eq:num_monomials}. Since these monomials are linearly independent it follows that all their coefficients in Equation \eqref{eq:lin_sym_fibre} equal zero. Lastly, all coefficients of the monomials composed of $u$ and $v$ are non-linear equations involving the unknown constants in the polynomial ans\"atze for $\eta_1(u,v)$ and $\eta_2(u,v)$, and hence the claim of the corollary follows.
\end{proof}
\dotfill\\
\hrule

\begin{remark}
If the degree of the polynomial reaction terms are $d_1=d_2=2$ and if we choose the degrees in the polynomial ans\"atze to be $d_{1,n}=d_{1,m}=d_{2,n}=d_{2,m}=2$ then the degree of the polynomial in Equation \eqref{eq:lin_sym_ansatz_modified} corresponding to the linearised symmetry condition is $\delta=2^6=64$ and the maximum number of equations is $D=65$. 
\end{remark}
\begin{remark}
  Note that the family of trivial generators are given by
  $$X_0=f(u,v)\left[\omega_1(u,v)\partial_u+\omega_2(u,v)\partial_v\right]$$
  where $f$ is an arbitrary function \cite{bluman1989symmetries}. Thus, by choosing the degrees $d_{1,n}>d_1$ and $d_{2,n}>d_2$ we are guaranteed to find trivial generators using this algorithm.
\end{remark}
\begin{remark}
  This algorithm generalises to linearised symmetry conditions with a higher number of states. If the number of states is $n$ and the degree $\delta$ is given by Equation \eqref{eq:num_monomials} then the number of monomials in each of the resulting polynomials is
  $${n+\delta-1\choose \delta}.$$
\end{remark}
\hrule
$\;$\\[0.5cm]
Subsequently, we will analyse the previously mentioned biological models starting with the LV-model. 


  