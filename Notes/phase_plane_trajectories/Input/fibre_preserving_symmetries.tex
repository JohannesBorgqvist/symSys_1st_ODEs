A common class of models in Mathematical Biology is that of two state time-invariant models. For example, these models typically describe the evolution of two competing populations or the evolution of two reacting proteins. To this end, we will now consider the following autonomous (specifically time-invariant) two state system of first order ODEs
\begin{equation}
  \begin{split}
    \dv{u}{\tau}&=\omega_1(u,v),\\
    \dv{v}{\tau}&=\omega_2(u,v).
  \end{split}
  \label{eq:sys_auto}
\end{equation}
where the right hand sides $\omega_1(u,v)$ and $\omega_2(u,v)$ referred to as the \textit{reaction terms} are independent of the time $\tau$. For these types of systems, there are two well-known symmetries. The first one is the time translation symmetry 

\begin{equation}
  \begin{split}
    \Gamma_{\epsilon}&:(\tau,u,v)\mapsto(\tau+\epsilon,u,v)\\
   \textrm{generated by}&\\
    X&=\partial_\tau
  \end{split}
  \label{eq:time_trans}
\end{equation}
and this symmetry is common to all autonomous models. The second symmetry is the \textit{trivial symmetry} generated by the infinitesimal generator of the Lie group given by
\begin{equation}
X=\partial_\tau+\omega_1(u,v)\partial_u+\omega_2(u,v)\partial_v
  \label{eq:trivial}
\end{equation}
and this vector field is parallel to the vector field of the original system of ODEs in Equation \eqref{eq:sys_auto}. Consequently, this symmetry maps points on a solution curve onto other points on the \textit{same solution curve}. Note that the infinitesimals in the time direction $\xi$ of these two generators are \textit{independent of the states} $u$ and $v$ meaning that these generators do not mix the time and state dependence. Such symmetries are called \textit{fibre preserving}, and next we are interested in such symmetries that are restricted to the $(u,v)$-phase plane.\\

\hrule
\begin{definition}[Fibre preserving symmetries of time-invariant models restricted to the phase plane]
Consider the autonomous system of ODEs in Equation \eqref{eq:sys_auto}. A \textit{fibre-preserving symmetry} $\Gamma_\epsilon$ that is restricted to the $(u,v)$-phase plane
is a symmetry that only acts on the fibre:
\begin{equation}
\Gamma_\epsilon:(\tau,u(\tau),v(\tau))\mapsto(\tau,\hat{u}(u,v;\epsilon),\hat{v}(u,v;\epsilon)).
  \label{eq:symmetry_phase_plane}
\end{equation}
Moreover, its infinitesimal generator of the Lie group $X$ lacks an infinitesimal in the $\tau$-direction, i.e. $\xi\equiv 0$, and it is given by
\begin{equation}
X = \eta_1(u,v)\partial_u+\eta_2(u,v)\partial_v.
  \label{eq:generator_phase_plane}
\end{equation}
\label{def:symmetry_fibre_phase_plane}
\end{definition}
\hrule
$\;$\\
\noindent Next, we demonstrate that the calculations of these fibre preserving symmetries of time-invariant models restricted to the phase plane are straightforward. This is due to the fact that the two linearised symmetry conditions in Equation \eqref{eq:lin_sym_ori}, are condensed into a single solvable PDE (Thm \ref{thm:lin_sym_cond_fibre_phase_plane}) in this case. 
\hrule
\begin{theorem}[The linearised symmetry condition of fibre preserving symmetries of time-invariant models restricted to the phase plane.]
  Consider the time-invariant system of ODEs in Equation \eqref{eq:sys_auto}. Further, let $\Gamma_\epsilon$ be a fibre preserving symmetry of this model that is restricted to the $(u,v)$-phase plane according to Definition \ref{def:symmetry_fibre_phase_plane} and let the corresponding infinitesimal generator of the Lie group $X$ be given by Equation \eqref{eq:generator_phase_plane}. Then, the infinitesimals $\eta_1(u,v)$ and $\eta_2(u,v)$ defining these symmetries solve the single PDE given by
\begin{equation}
 \omega_1^2 \pdv{\eta_2}{u}+\omega_1\omega_2\left(\pdv{\eta_2}{v}-\pdv{\eta_1}{u}\right)-\omega_2^2\pdv{\eta_1}{v}=\left(\pdv{\omega_2}{u}\omega_1-\omega_2\pdv{\omega_1}{u}\right)\eta_1+\left(\pdv{\omega_2}{v}\omega_1-\omega_2\pdv{\omega_1}{v}\right)\eta_2.
\label{eq:lin_sym_fibre}
\end{equation}

\label{thm:lin_sym_cond_fibre_phase_plane}
\end{theorem}
  \dotfill
\begin{proof}
  The dynamics in the $(u,v)$-phase plane of Equation \eqref{eq:sys_auto} is described by a single first order ODE:
\begin{equation}
\dv{v}{u}=\Omega(u,v)=\dfrac{\omega_2(u,v)}{\omega_1(u,v)}.
  \label{eq:sys_phase}
\end{equation}
Moreover, the total derivative for the phase plane is given by $D_u=\partial_u+(\dd v/\dd u)\partial_v$, and the linearised symmetry condition according to Equation \eqref{eq:lin_sym} is given by
\begin{equation}
  D_u\eta_2-\Omega D_u\eta_1=\pdv{\Omega}{u}\eta_1+\pdv{\Omega}{v}\eta_2\quad\textrm{whenever}\quad\dv{v}{u}=\Omega(u,v).
  \label{eq:lin_sym_phase}
  \end{equation}
The left hand side of the above equation can be written as:
\begin{equation*}
  D_u\eta_2-\Omega D_u\eta_1=\pdv{\eta_2}{u}+\Omega\left(\pdv{\eta_2}{v}-\pdv{\eta_1}{u}\right)-\Omega^2\pdv{\eta_1}{v}\quad\textrm{whenever}\quad\dv{v}{u}=\Omega(u,v).
\end{equation*}
The partial derivatives in the right hand side of Equation \eqref{eq:lin_sym_phase} are given by
\begin{align*}
  \pdv{\Omega}{u}&=\dfrac{\pdv{\omega_2}{u}\omega_1-\omega_2\pdv{\omega_1}{u}}{\omega_1^{2}},\\
  \pdv{\Omega}{v}&=\dfrac{\pdv{\omega_2}{v}\omega_1-\omega_2\pdv{\omega_1}{v}}{\omega_1^{2}}.
\end{align*}
By plugging in these partial derivatives into the right hand side of Equation \eqref{eq:lin_sym_phase}, equating the resulting expression with the left hand side and lastly multiplying the resulting equation with the factor $\omega_1^{2}$ results in the PDE in Equation \eqref{eq:lin_sym_fibre}. 
\end{proof}
\dotfill\\
\hrule
$\;$\\Morover, a subgroup of time-invariant models as in Equation \eqref{eq:sys_auto} that are common in mathematical biology are definied by polynomial reaction terms $\omega_1$ and $\omega_2$ respectively. In this case the linearised symmetry condition in Equation \eqref{eq:lin_sym_fibre} decomposes into a system of coupled first order PDEs where the number of equations depends on the degree of the reaction terms.\\
\hrule
\begin{corollary}[An algorithm for solving the linearised symmetry condition in the case of polynomial reaction terms.]
  Consider the time-invariant system of first order ODEs in Equation \eqref{eq:sys_auto} where the reaction terms $\omega_1(u,v)$ and $\omega_2(u,v)$ are two polynomials of degree $d_1$ and $d_2$ respectively. Furthermore, assume that two polynomial ans\"atze are used for the infinitesimals $\eta_1(u,v)$ and $\eta_2(u,v)$ where the degree of these ans\"atze are $d_3$ and $d_4$ respectively. Then, the linearised symmetry condition in Equation \eqref{eq:lin_sym_fibre} decomposes into a system of $d\in\mathbb{N}_+$ non-linear algebraic equations for the unknown coefficients in the polynomial ans\"atze for the infinitesimals $\eta_1(u,v)$ and $\eta_2(u,v)$ where the number of equations are bounded by $0\leq d\leq D$ where the upper bound is given by
  \begin{equation}
    D={2+\delta\choose \delta}=\dfrac{(2+\delta)!}{2!\delta!},\quad\delta=\max\left(2\,\max\left(d_1,d_2\right),d_3,d_4\right).
    \label{eq:num_monomials}
    \end{equation}

\label{thm:lin_sym_cond_fibre_phase_plane_polynomial}
\end{corollary}
\dotfill
\begin{proof}
In the case when the reaction terms $\omega_1(u,v)$ and $\omega_2(u,v)$ are polynomials of degree $d_1$ and $d_2$ respectively and where two polynomial ans\"atze of degree $d_3$ and $d_4$ are used for the infinitesimals $\eta_1(u,v)$ and $\eta_2(u,v)$, the linearised symmetry condition in Equation \eqref{eq:lin_sym_fibre} corresponds to finding the roots of a multivariate polynomial in two varibles, namely $u$ and $v$. The degree of this polynomial will either be determined by the term $\omega_1^2$, the term $\omega_2^2$ or the degrees $d_3$ and $d_4$. Therefore, the number of monomials composed of $u$ and $v$ in this polynomial is given by Equation \eqref{eq:num_monomials}. Since these monomials are linearly independent it follows that all their coefficients in Equation \eqref{eq:lin_sym_fibre} equal zero. Lastly, all coefficients of the monomials composed of $u$ and $v$ are non-linear equations involving the unknown constants in the polynomial ans\"atze for $\eta_1(u,v)$ and $\eta_2(u,v)$, and hence the claim of the corollary follows.
\end{proof}
\dotfill\\
\hrule

\begin{remark}
  Note that the family of trivial generators are given by
  $$X_0=f(u,v)\left[\omega_1(u,v)\partial_u+\omega_2(u,v)\partial_v\right]$$
  where $f$ is an arbitrary function \cite{bluman1989symmetries}. Thus, if we choose the degrees $d_3$ and $d_4$ to be larger than $\max\left(d_1,d_2\right)$, we are guaranteed to find trivial generators.
\end{remark}
\hrule
\noindent $\;$\\Subsequently, we will try to solve this linearised symmetry condition for the three models that were presented previously starting with the LV-model. 