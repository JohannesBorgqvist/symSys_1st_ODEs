The original SIR-model by Kermack-McKendrick can be formulated as follows describes how an infection is spread in a population consisting of three species: the number of susceptibles $S(t)$, the number of infectives $I(t)$ and the removed class $R(t)$ at time $t$. The model is given by

\begin{equation}
  \begin{split}
    \dv{S}{t}&=-rSI,\\
    \dv{I}{t}&=rSI-aI,\\
    \dv{R}{t}&=aI,\\
    \end{split}
  \label{eq:SIR}
\end{equation}
where $r>0$ is the infection rate and $a>0$ is the removal rate of infectives. Although this model looks like it has three states, qualitatively the state $R(t)$ can be viewed as a ``dummy state'' as the last ODE in Equation \eqref{eq:SIR} is independent of $R$. This means that we remove the third ODE without loosing any qualitative behaviour, and thereby reduce the above system to a two state system. From a symmetry perspective, the removal of the state $R$ would corresponds to $R$-autonomy implying that we have translations in $R$ in the same way as the model has a manifest time autonomy corresponding to translations in $t$. The qualitative behaviour of the SIR-model is restricted to the $SI$-phase plane.

The ODE describing the dynamics in the $SI$-phase plane is given by

\begin{equation}
\dv{I}{S}=\dfrac{rS-a}{rS}=\dfrac{a}{rS}-1
\label{eq:SI_phase_plane}
\end{equation}
and the reaction term can be considered separable with $g(I)=1$ and $f(S)=(rS-a)/rS$. As can be seen from Equation \eqref{eq:SI_phase_plane}, we have $I$-autonomy in the $SI$-phase plane, and by Theorem \ref{thm:separable} the non-trivial $S$-directional symmetry is given by $X=(rS/(rS-a))\partial_S$. In summary, the symmetries of the SIR-model are generated by the following infinitesimal generators of the Lie group
\begin{align} X_0&=\partial_t-rSI\partial_S+\left(rSI-aI\right)\partial_I+aI\partial_R\label{eq:SIR_0},\\
  X_t&=\partial_t\label{eq:SIR_t},\\
  X_S&=\left(\dfrac{rS}{rS-a}\right)\partial_S\label{eq:SIR_S},\\
  X_I&=\partial_I\label{eq:SIR_I},\\
  X_R&=\partial_R\label{eq:SIR_R}.
\end{align}
Here, the non-trivial generators $X_t$, $X_I$ and $X_R$ corresponds to translations. The symmetry generated by $X_S$ cannot be explicitly formulated, but we can write it down implicitly in terms of its canonical coordinates:
\begin{equation}
  \Gamma^{\mathrm{SIR},S}_{\epsilon}:(s,r_1,r_2,r_3)\mapsto(s+\epsilon,r_1,r_2,r_3),\quad s=S-\dfrac{\ln\left(a-rS\right)}{r},r_1=\tau,r_2=I,r_3=R,\\
  \label{eq:SIR_S_symmetry}
\end{equation}
and the transformed coordinate $\hat{S}$ can be found by numerically solving
\begin{equation}
  S-\dfrac{\ln\left(a-rS\right)}{r}+\epsilon=\hat{S}-\dfrac{\ln\left(a-r\hat{S}\right)}{r}
  \label{eq:SIR_S_symmetry_numerics}
\end{equation}
for given values of $S$ and $\epsilon$ respectively. Moreover, the solution trajectories in the $(I,S)$-phase plane are readily calculated by integrating the ODE in Equation \eqref{eq:SI_phase_plane} with respect to $S$ and these are given by the following equation \cite{murray2002}

\begin{equation}
  I=C-S+\dfrac{a}{r}\ln(S),\quad C=I_0+S_0-\dfrac{a}{r}\ln(S_0)
  \label{eq:SIR_trajectory}
\end{equation}
where the initial conditions in the $(I,S)$-phase plane are denoted by $I_0$ and $S_0$ respectively. The action of the $S$-directional symmetry $\Gamma^{\mathrm{SIR},s}_{\epsilon}$ in Equation \eqref{eq:SIR_S_symmetry} on the solution trajectories in Equation \eqref{eq:SIR_trajectory} is illustrated below (Fig \ref{fig:SIR_symmetry}). 


\begin{figure}[htbp!]
  \begin{center}
%\includegraphics[width=\textwidth]{SIR_symmetries}
\caption{\textit{The $S$-directional symmetry of the SIR-model}. }
\label{fig:SIR_symmetry}
\end{center}
\end{figure}